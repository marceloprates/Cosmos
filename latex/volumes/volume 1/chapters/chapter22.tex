
    \chapter{Prognostic Phenomena}
    is not found to be changed during the most intense Aurora ;but, on the other hand, the three expressions of the power ofterrestrial magnetism, declination, inclination, and intensity,are all affected by polar light, so that in the same night, andat different periods of the magnetic development, the sameend of the needle is both attracted and repelled. The.assertion made by Parry, on the strength of the data yielded byhis observations in the neighborhood of the magnetic pole atMelville Island, that the Aurora did not disturb, but ratherexercised a calming influence on the magnetic needle, has beensatisfactorily refuted by Parrys own more exact researches,detailed in his journal, and by the admirable observations ofRichardson, Hood, and Franklin in Northern Canada, andlastly by Bravais and Lottin in Lapland. The process of theAurora is, as has already been observed, the restoration of adisturbed condition of equilibrium. The effect on the needleis different according to the degree of intensity of the explosion. It was only unappreciable at the gloomy winter stationof Bosekop when the phenomenon of light was very faint andlow in the horizon. The shooting cylinders of rays have beenaptly compared to the flame which rises in the closed circuitof a voltaic pile between two points of carbon at a considerable distance apart, or, according to Fizeau, to the flame risingbetween a silver and a carbon point, and attracted or repelledby the magnet. This analogy certainly sets aside the necessity of assuming the existence of metallic vapors in the atmos.phere, which some celebrated physicists have regarded as thesubstratum of the northern light.

When we apply the indefinite term polar light to the luminous phenomenon which we ascribe to a galvanic current, thatis to say, to the motion of electricity in a closed circuit, wemerely indicate the local direction in which the evolution oflight is most frequently, although by no means invariably,seen. This phenomenon derives the greater part of its importance from the fact that the Earth becomes seflwminous,and that as a planet, besides the light which it receives fromthe central body, the Sun, it shows itself capable in itself ofdeveloping light. The intensity of the terrestrial light, or,rather, the luminosity which is diffused, exceeds, in cases ofthe brightest colored radiation toward the zenith, the lightof the Moon in its first quarter. Occasionally, as on the 7thof January, 1831, printed characters could be read withoutdifficulty. This almost uninterrupted development of light

 Kamtz, Lehrbuch der Meteorologie, bd. iii., s. 498 und 501.2

in the Karth leads us by analogy to the remarkable processexhibited in Venus. The portion of this planet which is notillumined by the Sun often shines with a phosphorescent lightof its own. It is not improbable that the Moon, Jupiter, andthe comets shine with an independent light, besides the reflected solar light visible through the polariscope. Withoutspeaking of the problematical but yet ordinary mode in whichthe sky is illuminated, when a low cloud may be seen to shinewith an uninterrupted flickering light for many minutes together, we still meet with other instances of terrestrial development of light in our atmosphere. In this category we mayreckon the celebrated luminous mists seen in 1783 and 1831 ; the steady luminous appearance exhibited without any flickering in great clouds observed by Rozier and Beccaria ; andlastly, as Arago well remarks, the faint diffused light whichguides the steps of the traveler in cloudy, starless, and moonless nights in autumn and winter, even when there is no snowon the ground. As in polar light or the electromagneticstorm, a current of brilliant and often colored light streamsthrough the atmosphere in high latitudes, so also in the torridzones between the tropics, the ocean simultaneously developslight over a space of many thousand square miles. Here thymagical effect of light is owing to the forces of organic nature.Foaming with light, the eddying waves flash in phosphorescent sparks over the wide expanse of waters, where every scintillation is the vital manifestation of an invisible animal world.So varied are the sources of terrestrial light Must we stillsuppose this light to be latent, and combined in vapors, inorder to explain Mosers images produced at a distanceadiscovery in which reality has hitherto manifested itself likea mere phantom of the imagination. 

. As the internal heat of our planet is connected on the onehand with the generation of electromagnetic currents andthe process of terrestrial light (a consequence of the magneticstorm), it, on the other hand, discloses to us the chief sourceof geognostic phenomena. We shall consider these in theirconnection with and their transition from merely dynamic disturbances, from the elevation of whole continents and mountam chains to the development and effusion of gaseous and

 Arago, on the dry fogs of 1783 and 1831, which illuminated thenight, in the Annuaire du Bureau des Longitudes, 1832, p.246 and 2503and, regarding extraordinary luminous appearances in clouds withoutstorms, see Notices sur la Tonnerre, in the Annuaire pour Van. 1838,p. 279285.

liquid fluids, of hot mud, and of those heated and moltenearths which become solidified into crystalline mineral masses.Modern geognosy, the mineral portion of terrestrial physics,has made no slight advance in having investigated this connection of phenomena. This investigation has led us awayfrom the delusive hypothesis, by which it was customary formerly to endeavor to explain, individually, every expression offorce in the terrestrial globe  it shows us the connection ofthe occurrence of heterogeneous substances with that whichonly appertains to changes in space (disturbances or elevations), and groups together phenomena which at first sightappeared most heterogeneous, as thermal springs, effusion ofcarbonic acid. and sulphurous vapor, innocuous salses (muderuptions), and the dreadful devastations of voleanic mountains. Jn a general view of nature, all these phenomena arefused together in one sole idea of the reaction of the interiorof a planet on its external surface. We thus recognize in thedepths of the earth, and in the increase of temperature withthe increase of depth from the surface, not only the germ ofdisturbing movements, but also of the gradual elevation ofwhole continents (as mountain chaizis on long fissures), of volcanic eruptions, and of the manifold production of mountainsand mineral masses. The influence of this reaction of theinterior on the exterior ig not, however, limited to inorganicnature alone. It is highly probable that, in an earlier world,more powerful emanations of carbonic acid gas, blended withthe atmosphere, must have increased the assimilation of carbon in vegetables, and that an inexhaustible supply of combustible matter (lignites and carboniferous formations) musthave been thus buried in the upper strata of the earth by therevolutions attending the destruction of vast tracts of forest.We likewise perceive that the destiny of mankind is in partdependent on the formation of the external surface of the earth,the direction of mountain tracts and high lands, and on thedistribution of elevated continents. It is thus granted to theinquiring mind to pass from link to link along the chain ofphenomena until it reaches the period when, in the solidifyingprocess of our planet, and in its first transition from the gaseous form to the agglomeration of matter, that portion of theinner heat of the Earth was developed, which does not belongto the action of the Sun.

 See Mantells Wonders of Geology, 1848, vol. i., p. 34, 36, 105;also Lyells Princip'es of Geology, vol. 1i.,and Daubeney On Volcanoes,2d ed., 1848, Part ii, ch. xxxil., xxxili.Tr.

Tn order to give a general delineation of the causal connection of geognostical phenomena, we will begin with thosewhose chief characteristic is dynamic, consisting in motionand in change in space. Earthquakes manifest themselvesby quick and successive vertical, or horizontal, or rotatory vibrations. In the very considerable number of earthquakeswhich I have experienced in both hemispheres, alike on landand at sea, the two firstnamed kinds of motion have often appeared to me to occur simultaneously. The minelike explosionthe vertical action from betow upwardwas most strikingly manifested in the overthrow of the town of Riobambain 1797, when the bodies of many of the inhabitants werefound to have been hurled to Cullea, a hill several hundredfeet in height, and on the opposite side of the River Lican.The propagation is most generally effected by undulations ina linear direction,t with a velocity of from twenty to twentyeight miles in a minute, but partly in circles of commotion orlarge ellipses, in which the vibrations are propagated withdecreasing intensity from a center toward the circumference.There are districts exposed to the action of two intersectingcircles of commotion. In Northern Asia, where the Fatherof History,t and subsequently Theophylactus Simocatta, described the districts of Scythia as free from earthquakes, Ihave observed the metalliferous portion of the Altai Mountains under the influence of a twofold focus of commotion, theLake of Baikal, and the voleano of the Celestial Mountain(Thianschan). When the circles of commotion intersect oneanotherwhen, for instance, an elevated plain lies betweentwo volcanoes simultaneously in a state of eruption, severalwavesystems may exist together, as in fluids, and not mutually disturb one another. We may even suppose interfer  See Daubeney On Voleanoes, 2d ed., 1848, p. 509.JTr.

 On the linear direction of earthquakes, see Daubeney On Voicanoes, p. 515.Tr.

t Herod, iv., 28. The prostration of the colossal statue of Memnon,which has been again restored (Letronne, La Statue Vocale de Memnon,1835, p. 25, 26), presents a fact in opposition to the ancient prejudicethat Egypt is free from earthquakes (Pliny, ii., 80); but the valley ofthe Nile does lie external to the circle of commotion of Byzantium, theArchipelago, and Syria (Ideler ad Aristot., Meteor., p. 584).

 SaintMartin, in the learned notes to Lebeau, Hist. du Bas Empire,t. ix., p. 401.

i Humboldt, Asie Centrale, t. ii., p. 110118. In regard to the difference between agitation of the surface and of the strata lying beneathit, see GayLussac, in the Annales de Chimie et de Physique, t. xxil.. p429.
    