
\chapter{Geognistic\\Phenomena}

\lettrine[lines=4]{\goudy A}{s} the internal heat of our planet is connected on the one hand with the generation of electromagnetic currents and the process of terrestrial light (a consequence of the magnetic storm), it, on the other hand, discloses to us the chief source of geognostic phenomena. We shall consider these in their connection with and their transition from merely dynamic disturbances, from the elevation of whole continents and mountain chains to the development and effusion of gaseous and liquid fluids, of hot mud, and of those heated and molten earths which become solidified into crystalline mineral masses. Modern geognosy, the mineral portion of terrestrial physics, has made no slight advance in having investigated this connection of phenomena. This investigation has led us away from the delusive hypothesis, by which it was customary formerly to endeavor to explain, individually, every expression of force in the terrestrial globe - it shows us the connection of the occurrence of heterogeneous substances with that which only appertains to changes in space (disturbances or elevations), and groups together phenomena which at first sight appeared most heterogeneous, as thermal springs, effusion of carbonic acid and sulphurous vapor, innocuous salses (mud eruptions), and the dreadful devastations of volcanic mountains.\footnote{See Mantell's Wonders of Geology, 1848, vol. i., p. 34, 36, 105; also Lyell's Principles of Geology, vol. ii., and Daubeney On Volcanoes, 2d ed., 1848, Part ii, ch. xxxii., xxxiii.} In a general view of nature, all these phenomena are fused together in one sole idea of the reaction of the interior of a planet on its external surface. We thus recognize in the depths of the earth, and in the increase of temperature with the increase of depth from the surface, not only the germ of disturbing movements, but also of the gradual elevation of whole continents (as mountain chains on long fissures), of volcanic eruptions, and of the manifold production of mountains and mineral masses. The influence of this reaction of the interior on the exterior is not, however, limited to inorganic nature alone. It is highly probable that, in an earlier world, more powerful emanations of carbonic acid gas, blended with the atmosphere, must have increased the assimilation of carbon in vegetables, and that an inexhaustible supply of combustible matter (lignites and carboniferous formations) must have been thus buried in the upper strata of the earth by the revolutions attending the destruction of vast tracts of forest. We likewise perceive that the destiny of mankind is in part dependent on the formation of the external surface of the earth, the direction of mountain tracts and highlands, and on the distribution of elevated continents. It is thus granted to the inquiring mind to pass from link to link along the chain of phenomena until it reaches the period when, in the solidifying process of our planet, and in its first transition from the gaseous form to the agglomeration of matter, that portion of the inner heat of the Earth was developed, which does not belong to the action of the Sun.