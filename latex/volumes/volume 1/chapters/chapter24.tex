\chapter{Hot Springs}


\lettrine[lines=4]{\goudy H}{ot} springs issue from the most various kinds of rocks. The hottest permanent springs that have hitherto been observed are, as my own researches confirm, at a distance from all volcanoes. I will here advert to a notice in my journal of the Aguas Calientes de las Trincheras, in South America, between Porto Cabello and Nueva Valencia, and the Aguas de Comangillas, in the Mexican territory, near Guanaxuato; the former of these, which issued from granite, had a temperature of 1945; the latter, issuing from basalt, 2055. The depth of the source from whence the water flowed with this temperature, judging from what we know of the law of the increase of heat in the interior of the earth, was probably 7140 feet, or above two miles. If the universally diffused terrestrial heat be the cause of thermal springs, as of active volcanoes, the rocks can only exert an influence by their different capacities for heat and by their conducting powers. The hottest of all permanent springs (between 203 and 209) are likewise, in a most remarkable degree, the purest, and such as hold in solution the smallest quantity of mineral substances. Their temperature appears, on the whole, to be less constant than that of springs between 122 and 165, which in Europe, at least, have maintained, in a most remarkable manner, their invariability of heat and mineral contents during the last fifty or sixty years, a period in which thermometrical measurements and chemical analyses have been applied with increased exactness. Boussingault found in 1823 that the thermal springs of Las Trincheras had risen 12 during the twenty-three years that had intervened since my travels in 1800.\footnote{Boussingault, in the Annals de Chimie, t. lii., p.181. The springof Chaudes Aigues, in Auvergne, is only 1769. It is also to be observed, that while the Aguas Calientes de las Trincheras, south of Porto Cabello (Venezuela), springing from granite cleft in regular beds, andfar from all volcanoes, have a temperature of fully 20696, all the springswhich rise in the vicinity of still active volcanoes (Pasto, Cotopaxi, andTunguragua) have a temperature of only $97^\circ$-$130^\circ$.} This calmly flowing spring is therefore now nearly 12 hotter than the intermittent fountains of the Geyser and the Strokr, whose temperature has recently been most carefully determined by Krug of Nidda. A very striking proof of the origin of hot springs by the sinking of cold meteoric water into the earth, and by its contact with a volcanic focus, is afforded by the volcano of Jorulla in Mexico, which was unknown before my American journey. When, in September, 1759, Jorullo was suddenly elevated into a mountain 1183 feet above the level of the surrounding plain, two small rivers, the Rio de Cuitimba and Rio de San Pedro, disappeared, and sometime afterward burst forth again, during violent shocks of an earthquake, as hot springs, whose temperature I found in 1803 to be $186^\circ . 4$.

The springs in Greece still evidently flow at the same places as in the times of Hellenic antiquity. The spring of Hrasinos, two hours journey to the south of Argos, on the declivity of Chaon, is mentioned by Herodotus. At Delphi we still see Cassotis (now the springs of St. Nicholas) rising south of the Lesche, and flowing beneath the Temple of Apollo; Castalia, at the foot of Phedriade; Pirene, near AcroCorinth; and the hot baths of Aidipsus, in Eubea, in which Sulla bathed during the Mithridatic war.\footnote{Cassotis (the spring of St. Nicholas) and Castalia, at the Phedriade, mentioned in Pausanias, x., 24, 25, and x., 8,9; Pirene (AcroCorinth) in Strabo, p. 379; the spring of Erasinos, at Mount Chaon, south of Argos, in Herod., vi., 67, and Pausanias, ii., 24, 7; the springs of Adipsusin Eubeea, some of which have a temperature of 88, while in others itranges between 144 and 1679, in Strabo, p. 60 and 447, and Atheneus,ij., 3,73; the hot springs of Thermopyle, at the foot of Eta, with atemperature of 149. All from manuscript notes by Professor Curtius,the learned companion of Otfried Miller.} I advert with pleasure to these facts, as they show us that, even in a country subject to frequent and violent shocks of earthquakes, the interior of our planet has retained for upward of 2000 years its ancient configuration in reference to the course of the open fissures that yield a passage to these waters. The Pontaine jaillissante of Lillers, in the Department des Pas de Calais, which was bored as early as the year 1126, still rises to the same height and yields the same quantity of water; and, as another instance, I may mention that the admirable geographer of the Caramanian coast, Captain Beaufort, saw in the district of Phaselis the same flame fed by emissions of inflammable gas which was described by Pliny as the flame of the Lycian Chimera.\footnote{Pliny, ii., 106; Seneca, Epist., 79,  3, ed. Ruhkopf (Beaufort, Survey of the Coast of Karamania, 1820, art. Yanar, near Deliktasch, theancient Phaselis, p. 24).. See, also, Ctesias, Fragm., cap. 10 p. 250,ed. Bahr; Strabo, lib. xiv,, p. 666, Casaub.;

[``Not far from the Deliktash, on the side of 2 mountain, is the perpetual fire described by Captain Beaufort. The travelers found it asbrilliant as ever, and even somewhat increased ; for, besides the largeflame in the corner of the ruins described by Beaufort, there were smalljets issuing from crevices in the side of the craterlike cavity five orsix feet deep. At the bottom was a shallow pool of sulphureous andturbid water, regarded by the Turks as a sovereign remedy for all skincomplaints. The soot deposited from the flames was regarded as efficacious for sore eyelids, and valued as a dye for the eyebrows.'' See the highly interesting and accurate work, Travels in Lycia, by Lieut.Spratt and Professor E. Forbes.] -- Tr.
}

The observation made by Arago in 1821, thatthe deepestArtesian wells are the warmest,\footnote{Arago, in the Annuaire pour 1835, p. 234.} threw great light on the origin of thermal springs, and on the establishment of the lawthat terrestrial heat increases with increasing depth. It isaremarkable fact, which has but recently been noticed, that atthe close of the third century, St. Patricius,\footnote{Acta 8. Patricti, p. 555, ed. Ruinart, t. ii., p. 385, Mazochi. Dareau de la Malle was the first to draw attention to this remarkable passage in the Recherches sur la Topographi de Carthage, 1835, p. 276(See, also, Seneca, Nat. Quest., iii., 24.)} probably Bishopof Pertusa, was led to adopt very correct views regarding thephenomenon of the hot springs at Carthage. On being askedwhat was the cause of boiling water bursting from the earth,he replied,  Fire is nourished in the clouds and in the interior of the earth, as Etna and other mountains near Naples mayteach you. The subterranean waters rise as if through siphons. The cause of hot springs is this waters which aremore remote from the subterranean fire are colder, while thosewhich rise nearer the fire are heated by it, and bring withthem to the surface which we inhabit an insupportable degreeof heat.

As earthquakes are often accompanied by eruptions of waterand vapors, we recognize in the Salses,\footnote{[True volcanoes, as we have seen, generate sulphureted hydrogenand muriatic acid, upheave tracts of land, and emit streams of meltedfeldspathic materials ; salses, on the contrary, disengage little else butcarbureted hydrogen, together with bitumen and other products of thedistillation of coal, and pour forth no other torrents except of mud, or argillaceous materials mixed up with water. Daubeney, op cit., p540.] -- Tr.} or small mud volcanoes, a transition from the changing phenomena presentedby these eruptions of vapor and thermal springs to the morepowerful and awful activity of the streams of lava that flowfrom volceanic mountains. If we consider these mountains assprings of molten earths producing voleanic rocks, we must remember that thermal waters, wheu impregnated with carbonicacid and sulphurous gases, are continually forming horizontally ranged strata of limestone (travertine) or conical elevations, as in Northern Africa (in Algeria), and in the Banosof Caxamarca, on the western declivity of the Peruvian Cordilleras. The travertine of Van Diemens Land (near HobartTown) contains, according to Charles Darwin, remains of avegetation that no longer exists. Lava and travertine, whichare constantly forming before our eyes, present us with thetwo extremes of geognostic relations.