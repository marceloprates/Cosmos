\chapter{Earthquakes}
    
\lettrine[lines=4]{\goudy I}{n} order to give a general delineation of the causal connection of geognostical phenomena, we will begin with those whose chief characteristic is dynamic, consisting in motion and in change in space. Earthquakes manifest themselves by quick and successive vertical, or horizontal, or rotatory vibrations.\footnote{[See Daubeney On Volcanoes, 2d ed., 1848, p. 509.] -- Tr.} In the very considerable number of earthquakes which I have experienced in both hemispheres, alike on land and at sea, the two first-named kinds of motion have often appeared to me to occur simultaneously. The mine-like explosion -- the vertical action from below upward -- was most strikingly manifested in the overthrow of the town of Riobamba in 1797, when the bodies of many of the inhabitants were found to have been hurled to Cullca, a hill several hundred feet in height, and on the opposite side of the River Lican. The propagation is most generally effected by undulations in a linear direction.\footnote{[On the linear direction of earthquakes, see Daubeney On Volcanoes, p. 515.] -- Tr.} with a velocity of from twenty to twenty-eight miles in a minute, but partly in circles of commotion or large ellipses, in which the vibrations are propagated with decreasing intensity from a center toward the circumference. There are districts exposed to the action of two intersecting circles of commotion. In Northern Asia, where the Father of History,\footnote{Herod, iv., 28. The prostration of the colossal statue of Memnon, which has been again restored (Letronne, La Statue Vocale de Memnon, 1835, p. 25, 26), presents a fact in opposition to the ancient prejudice that Egypt is free from earthquakes (Pliny, ii., 80); but the valley of the Nile does lie external to the circle of commotion of Byzantium, the Archipelago, and Syria (Ideler ad Aristot., Meteor., p. 584).} and subsequently Theophylactus Simocatta,\footnote{SaintMartin, in the learned notes to Lebeau, Hist. du Bas Empire, t. ix., p. 401.} described the districts of Scythia as free from earthquakes, I have observed the metalliferous portion of the Altai Mountains under the influence of a twofold focus of commotion, the Lake of Baikal, and the volcano of the Celestial Mountain (Thianschan).\footnote{Humboldt, Asie Centrale, t. ii., p. 110-118. In regard to the difference between agitation of the surface and of the strata lying beneath it, see Gay-Lussac, in the Annales de Chimie et de Physique, t. xxil., p. 429.} When the circles of commotion intersect one another - when, for instance, an elevated plain lies between two volcanoes simultaneously in a state of eruption, several wave systems may exist together, as in fluids, and not mutually disturb one another. We may even suppose interference to exist here, as in the intersecting waves of sound. The extent of the propagated waves of commotion will be increased on the upper surface of the earth, according to the general law of mechanics, by which, on the transmission of motion in elastic bodies, the stratum lying free on the one side endeavors to separate itself from the other strata.

Waves of commotion have been investigated by means of the pendulum and the seismometer\footnote{[This instrument, in its simplest form, consists of a basin filled with some viscid liquid, which, on the occurrence of a shock of an earthquake of sufficient force to disturb the equilibrium of the building in which it is placed, is tilted on one side, and the liquid made to rise in the same direction, thus showing by its height the degree of the disturbance. Professor J. Forbes has invented an instrument of this nature, although on a greatly improved plan. It consists of a vertical metal rod, having a ball of lead movable upon it. It is supported upon a cylindrical steel wire, which may be compressed at pleasure by means of a screw. A lateral movement, such as that of an earthquake, which carries forward the base of the instrument, can only act upon the ball through the medium of the elasticity of the wire, and the direction of the displacement will be indicated by the plane of vibration of the pendulum. A self-registering apparatus is attached to the machine. See Professor J. Forbes's account of his invention in Edinb Phil. Trans., vol. xv., Part i.] -- Tr.} with tolerable accuracy in respect to their direction and total intensity, but by no means with reference to the internal nature of their alternations and their periodic intumescenze. In the city of Quito, which lies at the foot of a still active volcano (the Rucu Pichincha), and at an elevation of 9540 feet above the level of the sea, which has beautiful cupolas, high vaulted churches, and massive edifices of several stories, I have often been astonished that the violence of the nocturnal earthquakes so seldom causes fissures in the walls, while in the Peruvian plains oscillations apparently much less intense injure low reed cottages. The natives, who have experienced many hundred earthquakes, believe that the difference depends less upon the length or shortness of the waves, and the slowness or rapidity of the horizontal vibrations,\footnote{Tutissimum est eum vibrat crispante dificiorum crepitu; et cum intumescit assurgens alternoque motu residet, innoxium et cum concur. rentia tecta contrario ictu arietant; quoniam alter motus alteri renititur. Undantis inclinatio et fluctus more quedam volutatio infesta est, aut cum in unam partem totus se motus impellit. Plin., ii., 82.} than on the uniformity of the motion in opposite directions. The circling rotatory commotions are the most uncommon, but, at the same time, the most dangerous. Walls were observed to be twisted, but not thrown down; rows of trees turned from their previous parallel direction; and fields covered with different kinds of plants found to be displaced in the great earthquake of Riobamba, in the province of Quito, on the 4th of February, 1797, and in that of Calabria, between the 5th of February and the 28th of March, 1782. The phenomenon of the inversion or displacement of fields and pieces of land, by which one is made to occupy the place of another, is connected with a translatory motion or penetration of separate terrestrial strata. When I made the plan of the ruined town of Riobamba, one particular spot was pointed out to me, where all the furniture of one house had been found under the ruins of another. The loose earth had evidently moved like a fluid in currents, which must be assumed to have been directed first downward, then horizontally, and lastly upward. It was found necessary to appeal to the Audencia, or Council of Justice, to decide upon the contentions that arose regarding the proprietorship of objects that had been removed to a distance of many hundred toises.

In countries where earthquakes are comparatively of much less frequent occurrence (as, for instance, in Southern Europe), a very general belief prevails, although unsupported by the authority of inductive reasoning,\footnote{Even in Italy they have begun to observe that earthquakes are unconnected with the state of the weather, that is to say, with the appearance of the heavens immediately before the shock. The numerical results of Friedrich Hoffmann (Hinterlassene Werke, bd. ii., 366375) exactly correspond with the experience of the Abbate Scina of Palermo. I have myself several times observed reddish clouds on the day of an earthquake, and shortly before it; on the 4th of November, 1799, I experienced two sharp shocks at the moment of a loud clap of thunder. (Relat. Hist., liv. iv., chap. 10.) The Turin physicist, Vassalli Eandi, observed Volta's electrometer to be strongly agitated during the protracted earthquake of Pignerol, which lasted from the 2d of April to the 17th of May, 1808; Journal de Physique, t. Ixvii., p. 291. But these indications presented by clouds, by modifications of atmospheric electricity, or by calms, can not be regarded as generally or necessarily connected with earthquakes, since in Quito, Peru, and Chili, as well as in Canada and Italy, many earthquakes are observed along with the purest and clearest skies, and with the freshest land and sea breezes. But if no meteorological phenomenon indicates the coming earthquake either on the morning of the shock or a few days previously, the influence of certain periods of the year (the vernal and autumnal equinoxes), the commencement of the rainy season in the tropics after long drought, and the change of the monsoons (according to general belief), can not be overlooked, even though the genetic connection of meteorological processes with those going on in the interior of our globe is still enveloped in obscurity. Numerical inquiries on the distribution of earthquakes throughout the course of the year, such as those of Von Hoff, Peter Merian, and Friedrich Hoffmarn, bear testimony to their frequency at the periods of the equinoxes. It is singular that Pliny, at the end of his fanciful theory of earthquakes, names the entire frightful phenomenon a subterranean storm; not so much in consequence of the rolling sound which frequently accompanies the shock, as because the elastic forces, concussive by their tension, accumulate in the interior of the earth when they are absent in the atmosphere. "Ventos in causa esse non dubium reor. Neque enim unquam intremiscunt terre, nisi sopito mari, coeloque adeo tranquillo, ut volatus avium non pendeant, subtracto omni spiritu qui vehit; nec unquam nisi post ventos conditos, scilicet in venas et cavernas ejus occulto afflatu. Neque aliud est in terra tremor, quam in nube tonitruum; nec hiatus aliud quam cum fulmen erumpit, incluso spiritu luctante et ad libertatem exire nitente." (Plin., ii., 79.) The germs of almost everything that has been observed or imagined on the causes of earthquakes, up to the present day, may be found in Seneca, Nat. Quest., vi., 431.} that a calm, an oppressive heat, and a misty horizon, are always the forerunners of this phenomenon. The fallacy of this popular opinion is not only refuted by my own experience, but likewise by the observations of all those who have lived many years in districts where, as in Cumana, Quito, Peru, and Chili, the earth is frequently and violently agitated. I have felt earthquakes in clear air and a fresh east wind, as well as in rain and thunderstorms. The regularity of the hourly changes in the declination of the magnetic needle and in the atmospheric pressure remained undisturbed between the tropics on the days when earthquakes occurred.\footnote{I have given proof that the course of the horary variations of thevarometer is not ae before or after earthquakes, in my Relat. Hist.,6 1, p. 311 and 513.} These facts agree with the observations made by Adolph Erman (in the temperate zone, on the 8th of March, 1829) on the occasion of an earthquake at Irkutsk, near the Lake of Baikal. During the violent earthquake of Cumana, on the 4th of November, 1799, I found the declination and the intensity of the magnetic force alike unchanged, but, to my surprise, the inclination of the needle was diminished about $48^\prime$.\footnote{Humboldt, Relat. Hist., t. i., p. 515-517.} There was no ground to suspect an error in the calculation, and yet, in the many other earthquakes which I have experienced on the elevated plateaus of Quito and Lima, the inclination as well as the other elements of terrestrial magnetism remained always unchanged. Although, in general, the processes at work within the interior of the earth may not be announced by any meteorological phenomena or any special appearance of the sky, it is, on the contrary, not improbable, as we shall soon see, that in cases of violent earthquakes some effect may be imparted to the atmosphere, in consequence of which they can not always act in a purely dynamic manner.

During the long-continued trembling of the ground in the Piedmontese valleys of Pelis and Clusson, the greatest changes in the electric tension of the atmosphere were observed while the sky was cloudless. The intensity of the hollow noise which generally accompanies an earthquake does not increase in the same degree as the force of the oscillations. I have ascertained with certainty that the great shock of the earthquake of Riobamba (4th Feb., 1797), one of the most fearful phenomena recorded in the physical history of our planet, was not accompanied by any noise whatever. The tremendous noise (el gran ruido) which was heard below the soil of the cities of Quito and Ibarra, but not at Tacunga and Hambato, nearer the center of the motion, occurred between eighteen and twenty minutes after the actual catastrophe. In the celebrated earthquake of Lima and Callao (28th of October, 1746), a noise resembling a subterranean thunderclap was heard at Truxillo a quarter of an hour after the shock, and unaccompanied by any trembling of the ground. In like manner, long after the great earthquake in New Granada, on the 16th of November, 1827, described by Boussingault, subterranean detonations were heard in the whole valley of Caucaduring twenty or thirty seconds, unattended by motion. The nature of the noise varies also very much, being either rolling, or rustling, or clanking like chains when moved, or like near thunder, as, for instance, in the city of Quito; or, lastly, clear and ringing, as if obsidian or some other vitrified masses were struck in subterranean cavities. As solid bodies are excellent conductors of sound, which is propagated in burned clay, for instance, ten or twelve times quicker than in the air, the subterranean noise may be heard at a great distance from the place where it has originated. In Caracas, in the grassy plains of Calabozo, and on the banks of the Rio Apure, which falls into the Orinoco, a tremendously loud noise, resembling thunder, was heard, unaccompanied by an earthquake, over a district of land 9200 square miles in extent, on the 30th of April, 1812, while at a distance of 632 miles to the northeast, the volcano of St. Vincent, in the small Antilles, poured forth a copious stream of lava. With respect to distance, this was as if an eruption of Vesuvius had been heard in the north of France. In the year 1744, on the great eruption of the volcano of Cotopaxi, subterranean noises, resembling the discharge of cannon, were heard in Honda, on the Magdalena River. The crater of Cotopaxi lies not only 18,000 feet higher than Honda, but these two points are separated by the colossal mountain chain of Quito, Pasto, and Popayan, no less than by numerous valleys and clefts, and they are 436 miles apart. The sound was certainly not propagated through the air, but through the earth, and at a great depth. During the violent earthquake of New Granada, in February, 1835, subterranean thunder was heard simultaneously at Popayan, Bogota, Santa Marta, and Caracas (where it continued for seven hours without any movement of the ground), in Haiti, Jamaica, and on the Lake of Nicaragua.

These phenomena of sound, when unattended by any perceptible shocks, produce a peculiarly deep impression even onpersons who have lived in countries where the earth has beenfrequently exposed to shocks. A striking and unparalleled instance of uninterrupted subterranean noise, unaccompanied byany trace of an earthquake, is the phenomenon known in theMexican elevated plateaux by the name of the  roaring andthe subterranean thunder (bramidos y truenos subterrancos)of Guanaxuato.\footnote{On the bramidos of Guanaxuato, see my Essai Polit. sur la Nouv.Espagne, t. i., p. 303. The subterranean noise, unaccompanied withany appreciable shock, in the deep mines and on the surface (the townof Guanaxuato lies 6830 feet above the level of the sea), was not heardin the neighboring elevated plains, but only in the mountainous partsof the Sierra, from the Cuesta de los Aguilares, near Marfil, to the northof Santa Rosa. There were individual parts of the Sierra 2428 milesnorthwest of Gaanaxuato, to the other side of Chichimequillo, near theboiling spring of San Jos\'{e} de Comangillas, to which the waves of sounddid not extend. Extremely stringent measures were adopted by themagistrates of the large mountain towns on the 14th of January, 1784,wheu the terror produced by these subterranean thunders was at itsheight. The flight of a wealthy family shall be punished with a fineof 1000 piasters, and that of a poor family with two months imprisonment. The militia shall bring back the fugitives. One of the mostremarkable points about the whole affair is the opinion which the magistrates (el cabildo) cherished of their own superior knowledge. Inone of their proclamas, I find the expression,  The magistrates, in theirwisdom (en su sabiduria), will at once know when there is actual danger, and will give orders for Hight; for the present, let processions beinstituted. The terror excited by the tremor gave rise to a famine,since it prevented the importation of corn from the tablelands, whereit abounded. The ancients were also aware that noises sometimes existed without earthquakes.Aristot., Meteor., ii., p. 802; Plin., ii., 80.Tho singular noise that was heard from March, 1822, to September,1824, in the Dalmatian island Meleda (sixteen miles from Ragusa), andon which Partsch has thrown much light, was occasionally accompaniedby shocks.} This celebrated and rich mountain citylies far removed from any active voleano. The noise beganabout midnight on the 9th of Januagy, 1784, and continuedfor a month. I have been enabled to give a circumstantial description of it from the report of many witnesses, and fromthe documents of the municipality, of which I was allowed tomake use. From the 13th to the 16th of January, it seemedto the inhabitants as if heavy clouds lay beneath their feet,from which issued alternate slow rolling sounds and short,quick claps of thunder. The noise abated as gradually as ithad begun. It was limited to a small space, and was notheard in a basaltic district at the distance of a few miles.Almost all the mhabitants, in terror, left the city, in whichlarge masses of silver ingots were stored ; but the most courageous, and those more accustomed to subterranean thunder,soon returned, in order to drive off the bands of robbers whohad attempted to possess themselves of the treasures of thecity. Neither on the surface of the earth, nor in mines 1600feet in depth, was the slightest shock to be perceived. Nosimilar noise had ever before been heard on the elevated tableland of Mexico, nor has this terrific phenomenon since occurredthere. Thus clefts are opened or closed in the interior of theearth, by which waves of sound penetrate to us or are impededin their propagation. 

The activity of an igneous mountain, however terrific andpicturesque the spectacle may be which it presents to our contemplation, is always limited to a very small space. It is farotherwise with earthquakes, which, although scarcely perceptible to the eye, nevertheless simultaneously propagate theirwaves to a distance of many thousand miles. The greatearthquake which destroyed the city of Lisbon on the Ist ofNovember, 1755, and whose effects were so admirably investigated by the distinguished philosopher Emmanuel Kant, wasfelt in the Alps, on the coast of Sweden, in the Antilles, Antigua, Barbadoes, and Martinique; in the great CanadianLakes, in Thuringia, in the flat country of Northern Germany, and in the small inland lakes on the shores of the Baltic.\footnote{[It has been computed that the shock of this earthquake pervadedan area of 700,000 miles, or the twelfth part of the circumference of theglobe. This dreadful shock lasted only five minutes it happened aboutnine oclock in the morning of the Feast of All Saints, when almost thewhole population was within the churches, owing to which circumstance no less than 30,000 persons perished by the fall of these edificea.See Daubeney On Volcanoes, p. 514517.] -- Tr.} Remote springs were interrupted in their flow, a phenomenon attending earthquakes which had been noticed amongthe ancients by Demetrius the Callatian. The hot springs ofT\'{e}plitz dried up, and returned, inundating every thing around,and having their waters colored with iron ocher. In Cadiz the sea rose to an elevation of sixtyfour feet, while in the Anlilles, where the tide usually rises only from twentysix totwentyeight inches, it suddenly rose above twenty feet, thewater being of an inky blackness. Jt has been computed thaton the 1st of November, 1755, a portion of the Earths surface, four times greater than that of Europe, was simultaneously shaken. As yet there is no manifestation of force knownto us, including even the murderous inventions of our ownrace, by which a greater number of people have been kilied inthe short space of a few minutes sixty thousand were destroyed in Sicily in 1693, from thirty to forty thousand in theearthquake of Riobamba in 1797, and probably five times asmany in Asia Minor and Syria, under Tiberius and Justinianthe elder, about the years 19 and 526.

There are instances in which the earth has been shaken for many successive days in the chain of the Andes in South America, but I am only acquainted with the following cases in which shocks that have been felt almost every hour for months together have occurred far from any volcano, as, for instance, on the eastern declivity of the Alpine chain of Mount Cenis, at Fenestrelles and Pignerol, from April, 1808; between New Madrid and Little Prairic,\footnote{Drake, Nat. and Statist. View of Cincinnati, p. 232238; Mitchell,in the Transactions of the Lit. and Philos. Soc. of New York, vol. i., p.281308. In the Piedmontese county of Pignerol, glasses of water, filledty the very brim, 2xhibited for hours a continuous motion.} north of Cincinnati, in the United States of America, in December, 1811, as well as through the whole winter of 1812; and in the Pachalik of Aleppo, in the months of August and September, 1822. As the mass of the people are seldom able to rise to general views, and are consequently always disposed to ascribe great phenomena to local telluric and atmospheric processes, wherever the shaking of the earth is continued for a long time, fears of the eruption of a new volcano are awakened. In some few cases, this apprehension has certainly proved to be well grounded, as, for instance, in the sudden elevation of volcanic islands, and as we see in the elevation of the volcano of Jorullo, a mountain elevated 1684 feet above the ancient level of the neighboring plain, on the 29th of September, 1759, after ninety days of earthquake and subterranean thunder.

If we could obtain information regarding the daily condition of all the earth's surface, we would probably discover that the earth is almost always undergoing shocks at some point of its surface, and is continually influenced by the reaction of the interior on the exterior. The frequency and general prevalence of a phenomenon which is probably dependent on the raised temperature of the deepest molten strata explain its independence of the nature of the mineral masses in which it manifests itself. Earthquakes have even been felt in the loose alluvial strata of Holland, as in the neighborhood of Middleburg and Vliessingen on the 23rd of February, 1828. Granite and mica slate are shaken as well as limestone and sandstone, or as trachyte and amygdaloid. It is not, therefore, the chemical nature of the constituents, but rather the mechanical structure of the rocks, which modifies the propagation of the motion, the wave of commotion. Where this wave proceeds along a coast, or at the foot and in the direction of a mountain chain, interruptions at certain points have sometimes been remarked, which manifested themselves during the course of many centuries. The undulation advances in the depths below, but is never felt at the same points on the surface. The Peruvians\footnote{In Spanish they say, "rocas que hacen puente". With this phenomenon of non-propagation through superior strata is connected the remarkable fact that in the beginning of this century shocks were felt in the deep silver mines at Marienberg, in the Saxony mining district, while not the slightest trace was perceptible at the surface. The miners ascended in a state of alarm. Conversely, the workmen in the mines of Falun and Persberg felt nothing of the shocks which in November, 1823, spread dismay among the inhabitants above ground.} say of these unmoved upper strata that they form a bridge. As the mountain chains appear to be raised on fissures, the walls of the cavities may perhaps favor the direction of undulations parallel to them; occasionally, however, the waves of commotion intersect several chains almost perpendicularly. Thus we see them simultaneously breaking through the littoral chain of Venezuela and the Sierra Parime. In Asia, shocks of earthquakes have been propagated from Lahore and from the foot of the Himalaya (22nd of January, 1832) transversely across the chain of the Hindoo Chou to Badakschan, the upper Oxus, and even to Bokhara.\footnote{Sir Alex. Burnes, Travels in Bokhara, vol. i., p. 18; and Wathen Mem. on the Usbek State, in the Journ l of the Asiatic Society of Bengai, vol. iii., p. 337.} The circles of commotion unfortunately expand occasionally in consequence of a single and unusually violent earthquake. It is only since the destruction of Cumana, on the 14th of December, 1797, that shocks on the southern coast have been felt in the mica slate rocks of the peninsula of Maniquarez, situated opposite to the chalk hills of the mainland. The advance from south to north was very striking in the almost uninterrupted undulations of the soil in the alluvial valleys of the Mississippi, the Arkansas, and the Ohio, from 1811 to 1813. It seemed here as if subterranean obstacles were gradually overcome, and that the way being once opened, the undulatory movement could be freely propagated.

Although earthquakes appear at first sight to be simply dynamic phenomena of motion, we yet discover, from well-attested facts, that they are not only able to elevate a whole district above its ancient level (as, for instance, the Ulla Bund, after the earthquake of Cutch, in June, 1819, east of the Delta of the Indus, or the coast of Chili, in November, 1822), but we also find that various substances have been ejected during the earthquake, such as hot water at Catania in 1818; hot steam at New Madrid, in the Valley of the Mississippi, in 1812; irrespirable gases, Mofettes, which injured the flocks grazing in the chain of the Andes; mud, black smoke, and even flames, at Messina in 1781, and at Cumana on the 14th of November, 1797. During the great earthquake of Lisbon, on the 1st of November, 1755, flames and columns of smoke were seen to rise from a newly-formed fissure in the rock of Alvidras, near the city. The smoke in this case became more dense as the subterranean noise increased in intensity.\footnote{Philos. Transact, vol. xlix. p. 414.} At the destruction of Riobamba, in the year 1797, when the shocks were not attended by any outbreak of the neighboring volcano, a singular mass called the Moya was uplifted from the earth in numerous continuous conical elevations, the whole being composed of carbon, crystals of augite, and the silicious shields of infusoria. The eruption of carbonic acid gas from fissures in the Valley of the Magdalene, during the earthquake of New Granada, on the 16th of November, 1827, suffocated many snakes, rats, and other animals. Sudden changes of weather, such as the occurrence of the rainy season in the tropics, at an unusual period of the year, have sometimes succeeded violent earthquakes in Quito and Peru. Do gaseous fluids rise from the interior of the earth and mix with the atmosphere, or are these meteorological processes the action of atmospheric electricity disturbed by the earthquake? In the tropical regions of America, where sometimes not a drop of rain falls for ten months together, the natives consider the repeated shocks of earthquakes, which do not endanger the low reed huts, as auspicious harbingers of fruitfulness and abundant rain.

The intimate connection of the phenomena which we have considered is still hidden in obscurity. Elastic fluids are doubtlessly the cause of the slight and perfectly harmless trembling of the earth's surface, which has often continued several days (as in 1816, at Scaccia, in Sicily, before the volcanic elevation of the island of Julia), as well as of the terrific explosion accompanied by loud noise. The focus of this destructive agent, the seat of the moving force, lies far below the earth's surface; but we know as little of the extent of this depth as we know of the chemical nature of these vapors that are so highly compressed. At the edges of two craters, Vesuvius, and the towering rock which projects beyond the great abyss of Pichincha, near Quito, I have felt periodic and very regular shocks of earthquakes, on each occasion from 20 to 30 seconds before the burning scoria or gases were erupted. The intensity of the shocks was increased in proportion to the time intervening between them, and, consequently, to the length of time in which the vapors were accumulating. This simple fact, which has been attested by the evidence of so many travelers, furnishes us with a general solution of the phenomenon, in showing that active volcanoes are to be considered as safety valves for the immediate neighborhood. The danger of earthquakes increases when the openings of the volcano are closed and deprived of free communication with the atmosphere; but the destruction of Lisbon, of Caracas, of Lima, of Cashmir in 1554,\footnote{On the frequency of earthquakes in Cashmir, see Troyers German translation of the ancient Radjataringini, vol. ii., p. 297, and Carl Hiigel, Reisen, bd. ii., 8. 184.} and of so many cities of Calabria, Syria, and Asia Minor, shows us, on the whole, that the force of the shock is not the greatest in the neighborhood of active volcanoes.

As the impeded activity of the volcano acts upon the shocks of the earth's surface, so do the latter react on the volcanic phenomena. Openings of fissures favor the rising of cones of eruption, and the processes which take place in these cones, by forming a free communication with the atmosphere. A column of smoke, which had been observed to rise for months together from the volcano of Pasto, in South America, suddenly disappeared when, on the 4th of February, 1797, the province of Quito, situated at a distance of 192 miles to the south, suffered from the great earthquake of Riobamba. After the earth had continued to tremble for some time throughout the whole of Syria, in the Cyclades, and in Eubeea, the shocks suddenly ceased on the eruption of a stream of hot mud on the Lelantine plains near Chalcis.\footnote{Strabo, lib. i., p. 100, Casaub. That the expression 77/00 diant-
pov morauéy does not mean erupted mud, but lava, is obvious from a
passage in Strabo, lib. vi., p. 412. Compare Walter, in his Abnakme der
Vulkanischen Thatigheit in Historischen Zeiten (On the Decrease of Vol-
canic Activity during Historical Times), 1844, s. 25.} The intelligent geographer of Amasea, to whom we are indebted for the notice of this circumstance, further remarks: ``Since the craters of Etna have been opened, which yield a passage to the escape of fire, and since burning masses and water have been ejected, the country near the sea-shore has not been so much shaken as at the time previous to the separation of Sicily from Lower Italy, when all communications with the external surface were closed.''

We thus recognize in earthquakes the existence of a volcanic force, which, although everywhere manifested, and as generally diffused as the internal heat of our planet, attains but rarely, and then only at separate points, sufficient intensity to exhibit the phenomenon of eruptions. The formation of veins, that is to say, the filling up of fissures with crystalline masses bursting forth from the interior (as basalt, melaphyre, and greenstone), gradually disturbs the free intercommunication of elastic vapors. This tension acts in three different ways, either in causing disruptions, or sudden and retroversed elevations, or, finally, as was first observed in a great part of Sweden, in producing changes in the relative level of the sea and land, which, although continuous, are only appreciable at intervals of long period.

Before we leave the important phenomena which we have considered, not so much in their individual characteristics as in their general physical and geognostical relations, I would advert to the deep and peculiar impression left on the mind by the first earthquake which we experience, even where it is not attended by any subterranean noise.\footnote{[Dr. Tschudi, in his interesting work, Travels in Peru, translated from the German by Thomasina Ross, p. 170, 1847, describes strikingly the effect of an earthquake upon the native and upon the stranger. “No familiarity with the phenomenon can blunt this feeling. The inhabitant of Lima, who from childhood has frequently witnessed these convulsions of nature, is roused from his sleep by the shock, and rushes from his apartment with the cry of Misericordia! The foreigner from the north of Europe, who knows nothing of earthquakes but by description, waits with impatience to feel the movement of the earth, and longs to hear with his own ear the subterranean sounds which he has hitherto considered fabulous. With levity he treats the apprehension of a coming convulsion, and laughs at the fears of the natives; but, as soon as his wish is gratified, he is terror-stricken, and is involuntarily prompted to seek safety in flight.”] -- Tr.} This impression is not, in my opinion, the result of a recollection of those fearful pictures of devastation presented to our imaginations by the historical narratives of the past, but - is rather due to the sudden revelation of the delusive nature of the inherent faith by which we had clung to a belief in the immobility of the solid parts of the earth. We are accustomed from early childhood to draw a contrast between the mobility of water and the immobility of the soil on which we tread; and this feeling is confirmed by the evidence of our senses. When, therefore, we suddenly feel the ground move beneath us, a mysterious and natural force, with which we are previously unacquainted, is revealed to us as an active disturbance of stability. A moment destroys the illusion of a whole life; our deceptive faith in the repose of nature vanishes, and we feel transported, as it were, into a realm of unknown destructive forces. Every sound - the faintest motion in the air - arrests our attention, and we no longer trust the ground on which we stand. Animals, especially dogs and swine, participate in the same anxious disquietude; and even the crocodiles of the Orinoco, which are at other times as dumb as our little lizards, leave the trembling bed of the river and run with loud cries into the adjacent forests.

To man, the earthquake conveys an idea of some universal and unlimited danger. We may flee from the crater of a volcano in active eruption or from the dwelling whose destruction is threatened by the approach of the lava stream, but in an earthquake, direct our flight whithersoever we will, we still feel as if we trod upon the very focus of destruction. This condition of the mind is not of long duration, although it takes its origin in the deepest recesses of our nature, and when a series of faint shocks succeed one another, the inhabitants of the country soon lose every trace of fear. On the coasts of Peru, where rain and hail are unknown, no less than the rolling thunder and the flashing lightning, these luminous explosions of the atmosphere are replaced by the subterranean noises which accompany earthquakes.\footnote{[``Along the whole coast of Peru, the atmosphere is almost uniformly in a state of repose. It is not illuminated by the lightning's flash, or disturbed by the roar of the thunder; no deluges of rain, no fierce hurricanes, destroy the fruits of the fields, and with them the hopes of the husbandman. But the mildness of the elements above ground is frightfully counterbalanced by their subterranean fury. Lima is frequently visited by earthquakes, and several times the city has been reduced to a mass of ruins. On average, forty-five shocks may be counted on in the year. Most of them occur in the latter part of October, November, December, January, May, and June. Experience gives reason to expect the visitation of two desolating earthquakes in a century. The period between the two is from forty to sixty years. The most considerable catastrophes experienced in Lima since Europeans have visited the west coast of South America happened in the years 1586, 1630, 1687, 1713, 1746, 1806. There is reason to fear that in the course of a few years this city may be the prey of another such visitation.'' —Tschudi, op. cit.] -- Tr.} Long habit, and the very prevalent opinion that dangerous shocks are only to be apprehended two or three times in the course of a century, cause faint oscillations of the soil to be regarded in Lima with scarcely more attention than a hail storm in the temperate zone.