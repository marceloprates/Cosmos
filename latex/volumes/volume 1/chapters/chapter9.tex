
\chapter[Translatory Motion of ...]{Translatory Motion\\of The Solar System}

\lettrine[lines=4]{\goudy W}{e} have hitherto considered that which belongs to our solar system, that world of material forms governed by the Sun, which includes the primary and secondary planets, comets of short and long periods of revolution, meteoric asteroids, which move thronged together in streams, either sporadically or in closed rings, and finally a luminous nebulous ring that revolves around the Sun in the vicinity of the Earth, and for which, owing to its position, we may retain the name of zodiacal light. Everywhere the law of periodicity governs the motions of these bodies, however different may be the amount of tangential velocity or the quantity of their agglomerated material parts; the meteoric asteroids which enter our atmosphere from the external regions of universal space are alone arrested in the course of their planetary revolution and retained within the sphere of a larger planet. In the solar system, whose boundaries determine the attractive force of the central body, comets are made to revolve in their elliptical orbits at a distance 44 times greater than that of Uranus; nay, in those comets whose nucleus appears to us, from its inconsiderable mass, like a mere passing cosmical cloud, the Sun exercises its attractive force on the outermost parts of the emanations radiating from the tail over a space of many millions of miles. Central forces, therefore, at once constitute and maintain the system.

Our Sun may be considered as at rest when compared to all the large and small, dense and almost vaporous cosmic bodies that appertain to and revolve around it; but it actually rotates round the common center of gravity of the whole system, which occasionally falls within itself, that is to say, remains within the material circumference of the Sun, whatever changes may be assumed by the positions of the planets. A very different phenomenon is that presented by the translatory motion of the Sun, that is, the progressive motion of the center of gravity of the whole solar system in universal space. Its velocity is such\footnote{Bessel, in Schum., Jahrb. fur 1839, 8.51; probably four millionsof miles daily, in a relative velocity of at the least 3,336,000 miles, ormore than double the velocity of revolution of the Earth in her orbitsound the Sun.} that, according to Bessel, the relative motion of the Sun, and that of 61 Cygni, is not less in one day than 3,336,000 geographical miles. This change of the entire solar system would remain unknown to us, if the admirable exactness of our astronomical instruments of measurement, and the advancement recently made in the art of observing, did not cause our advance toward remote stars to be perceptible, like an approximation to the objects of a distant shore in apparent motion. The proper motion of the star 61 Cygni, for instance, is so considerable, that it has amounted to a whole degree in the course of 700 years.

The amount or quantity of these alterations in the fixedstars (that is to say, the changesin the relative position ofselfluminous stars toward each other), can be determinedwith a greater degree of certainty than we are able to attachto the genetie explanation of the phenomenon. After takinginto consideration what is due to the precession of the equinoxes, and the nutation of the earths axis produced by theaction of the Sun and Moon on the spheroidal figure of ourglobe, and what may be ascribed to the transmission of light,that is to say, to its aberration, and to the parallax formed bythe diametrically opposite position of the Earth in its courseround the Sun, we still find that there is a residual portion of the annual motion of the fixed stars due to the translationof the whole solar system in universal space, and to the trueproper motion of the stars. The difficult problem of numerically separating these two elements, the true and the apparent motion, has beeneflected by the careful study of the direction of the motion of certain individual stars, and by theconsideration of the fact that, if all the stars were in.a stateof absolute rest, they would appear perspectively to recedefrom the point in space toward which the Sun was directingits course. But the ultimate result of this investigation, confirmed by the calculus of probabilities, is, that our solar system and the stars both change their places in space. According to the admirable researches of Argelander at Abo, whohas extended and more perfectly developed the work begun byWilliam Herschel and Prevost, the Sun moves in the direction of the constellation Hercules, and probably, from thecombination of the observations made of 537 stars, toward apoint lying (at the equinox of 17925) at 257 497 R.A., and28 497 N.D. It is extremely difficult, in investigations ofthis nature, to separate the absolute from the relative motion,and to determine what is alone owing to the solar system.\footnote{Regarding the motion of the solar system, according to Bradley,Tobias Mayer, Lambert, Lalande, and William Herschel, see Arago, inthe Annuaire, 1842, p. 388399 ; Argelander, in Schum., Astron. Nachr.,No. 363, 364, 398, and in the treatise Von der eigenen Bewegung desSonnensystems (On the proper Motion of the Solar em, 1837, s. 43,respecting Perseus as the central body of the whole stellar stratum,likewise Otho Struve, in the Bull. del Acad. de St. P\'{e}tersb., 1842, t. x.,No. 9, p. 187139. The lastnamed astronomer has found, by a morerecent combination, 261 23 R.A.  37 36 Decl. for the direction ofthe Suns motion; and, taking the mean of his own results with that ofArgelander, we have, by a combination of 797 stars, the formula 259Y R.A.  34 36 Decl.}

If we consider the proper, and not the perspective motions of the stars, we shall find many that appear to be distributed in groups, having an opposite direction; and facts hitherto observed do not, at any rate, render it a necessary assumption that all parts of our starry stratum, or the whole of the stellar islands filling space, should move round one large unknown luminous or nonluminous central body. The tendency of the human mind to investigate ultimate and highest causes certainly inclines the intellectual activity, no less than the imagination of mankind, to adopt such an hypothesis. Even the Stagirite proclaimed that every thing which is moved must be referable to a motor, and that there would be no end to the concatenation of causes if there were not one primordial immovable motor.\footnote{Aristot., de Calo, iii., 2, p. 301, Bekker; Phys., viii., 5, p. 256.}

The manifold translatory changes of the stars, not those produced by the parallaxes at which they are seen from the changing position of the spectator, but the true changes constantly going on in the regions of space, afford us incontrovertible evidence of the dominion of the laws of attraction in the remotest regions of space, beyond the limits of our solar system. The existence of these laws is revealed to us by many phenomena, as, for instance, by the motion of double stars, and by the amount of retarded or accelerated motion in different parts of their elliptic orbits. Human inquiry need no longer pursue this subject in the domain of vague conjecture, or amid the undefined analogies of the ideal world; for even here the progress made in the method of astronomical observations and calculations has enabled astronomy to take up its position on a firm basis. It is not only the discovery of the astounding numbers of double and multiple stars revolving round a center of gravity lying outside their system (2800 such systems having been discovered up to 1837), but rather the extension of our knowledge regarding the fundamental forces of the whole material world, and the proofs we have obtained of the universal empire of the laws of attraction, that must be ranked among the most brilliant discoveries of the age. The periods of revolution of colored stars present the greatest differences; thus, in some instances, the period extends to 43 years, as in 7 of Corona, and in others to several thousands, as in 66 of Cetus, 38 of Gemini, and 100 of Pisces. Since Herschel's measurements in 1782, the satellite of the nearest star in the triple system of  of Cancer has completed more than one entire revolution. By a skillful combination of the altered distances and angles of position,\footnote{Savary, in the Connaissance des T'ems, 1830, p. 56 and 163. Encke,Berl. Jahrb., 1832, 8. 253, c. Arago, in the Annuaire, 1834, p. 260,295. John Herschel, in the Memoirs of the Astronom. Soc., vol. v., p. 171.} the elements of these orbits may be found, conclusions drawn regarding the absolute distance of the double stars from the Earth, and comparisons made between their mass and that of the Sun. Whether, however, here and in our solar system, quantity of matter is the only standard of the amount of attractive force, or whether specific forces of attraction proportionate to the mass may not at the same time come into operation, as Bessel was the first to conjecture, are questions whose practical solution must be left to future ages.\footnote{Bessel, Untersuchung. des Theils der planetarischen Storungen,welche aus der Bewegumg der Sonne entstehen (An Investigation of the portion of the Planetary Disturbances depending on the Motion of theSun) in Abk. der Berl. Akad. der Wissensch., 1824 (Mathem. Classe),s. 26. The question has been raised by John Tobias Mayer, in Comment. Soc. Reg. G\'{e}tting., 18041808, vol. xvi., p. 3168.} Wherwe compare our Sun with the other fixed stars, that is, with other selfuminous Suns in the lenticular starry stratum of whichour system forms a part, we find, at least in the case of some,that channels are opened to us, which may lead, at all events,to an approximate and limited knowledge of their relativedistances, volumes, and masses, and of the velocities of theirtranslatory motion. If we assume the distance of Uranusfrom the Sun to be nineteen times that of the Earth, that isto say, nineteen times as great as that of the Sun from theEarth, the central bedy of our planetary system will be 11,900times the distance of Uranus from the star  in the constellation Centaur, almost 31,300 from 61 Cygni, and 41,600 fromVega in the constellation Lyra. The comparison of the volume of the Sun with that of the fixed stars of the first magnitude is dependent upon the apparent diameter of the latterbodiesan extremely uncertain optical element. If even weassume, with Herschel, that the apparent diameter of Arcturus is only a tenth part of a second, it still follows that thetrue diameter of this star is eleven times greater than that ofthe Sun.\footnote{Philos. Trans. for 1803, p. 225. Arago, in the Annuaire, 1842, p.375. In order to obtain a clearer idea of the distances ascribed in arather earlier part of the text to the fixed stars, let us assume that theEarth is a distance of one foot from the Sun; Uranus is then 19 feet,and Vega Lyra is 158 geographical miles from it.} The distance of the star 61 Cygni, made knownby Bessel, has led approximately to a knowledge of the quantity of matter contained in this body as a double star. Notwithstanding that, since Bradleys observations, the portionof the apparent orbit traversed by this star is not sufficientlygreat to admit of our arriving with perfect exactness at thetrue orbit and the major axis of this star, it has been conjectured with much probability by the great Konigsberg astronomer,\footnote{Bessel, in Schum., Jahrb., 1839, s. 53.} that the mass of this double star can not be very considerably larger or smaller than half of the mass of the Sun.This result is from actual measurement. The analogies deduced from the relatively larger mass of those planets in oursolar system that are attended by satellites, and from the factthat Struve has discovered six times more double stars among the brighter than among the telescopic fixed stars, have ledother astronomers to conjecture that the average mass of thelarger number of the binary stars exceeds the mass of theSun.\footnote{Madler, Astron., 8. 476; also in Schum., Jahrb., 1839, 8 95.} We are, however, far from having arrived at generalresults regarding this subject. Our Sun, according to Argelander, belongs, with reference to proper motion in space, tothe class of rapidlymoving fixed stars.

The aspect of the starry heavens, the relative position of stars and nebula, the distribution of their luminous masses, the picturesque beauty, if I may so express myself, of the whole firmament, depend in the course of ages conjointly upon the proper motion of the stars and nebula, the translation of our solar system in space, the appearance of new stars, and the disappearance or sudden diminution in the intensity of the light of others, and, lastly and specially, on the changes which the Earth's axis experiences from the attraction of the Sun and Moon. The beautiful stars in the constellation of the Centaur and the Southern Cross will at some future time be visible in our northern latitudes, while other stars, as Sirius and the stars in the Belt of Orion, will in their turn disappear below the horizon. The places of the North Pole will successively be indicated by the stars 3 and a Cephei, and 6 Cygni, until after a period of 12,000 years, Vega in Lyra will shine forth as the brightest of all possible pole stars. These data give us some idea of the extent of the motions which, divided into infinitely small portions of time, proceed without intermission in the great chronometer of the universe. If for a moment we could yield to the power of fancy, and imagine the acuteness of our visual organs to be made equal with the extremest bounds of telescopic vision, and bring together that which is now divided by long periods of time, the apparent rest that reigns in space would suddenly disappear. We should see the countless host of fixed stars moving in thronged groups in different directions; nebulae wandering through space, and becoming condensed and dissolved like cosmical clouds; the veil of the Milky Way separated and broken up in many parts, and motion ruling supreme in every portion of the vault of heaven, even as on the Earth's surface, where we see it unfolded in the germ, the leaf, and the blossom, the organisms of the vegetable world. The celebrated Spanish botanist Cavanilles was the first who entertained the idea of seeing grass grow, and he directed the horizontal micrometer threads of a powerfully magnifying glass at one time to the apex of the shoot of a bambusa, and at another on the rapidly growing stem of an American aloe (Agave Americana), precisely as the astronomer places his cross of network against a culminating star. In the collective life of physical nature, in the organic as in the sidereal world, all things that have been, that are, and will be, are alike dependent on motion.