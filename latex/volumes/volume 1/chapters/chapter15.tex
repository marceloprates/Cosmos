
\chapter{Density of the Earth}

\lettrine[lines=4]{\goudy W}{hen} the Earth had been measured, it still had to be weighed. The oscillations of the pendulum\footnotemark and the plummet have here likewise served to determine the mean density of the Earth, either in connection with astronomical and geodetic operations, with the view of finding the deflection of the plummet from a vertical line in the vicinity of a mountain, or by a comparison of the length of the pendulum in a plain and on the summit of an elevation, or, finally, by the employment\footnotetext{La Cailles pendulum measurements at the Cape of Good Hope, which have been calculated with much care by Mathieu (Delambre, Hist. de U Astron. au 18me Siècle, p. 479), give a compression of szy.qth; but, from several comparisons of observations made in equal latitudes in the two hemispheres (New Holland and the Malouines (Falkland Islands), compared with Barcelona, New York, and Dunkirk), there is as yet no reason for supposing that the mean compression of the southern hemisphere is greater than that of the northern. (Biot, in the Mém. del Acad. des Sciences, t. viii., 1829, p. 3941.} of a torsion balance, which may be considered as a horizontally vibrating pendulum for the measurement of the relative density of neighboring strata. Of these three methods\footnote{The three methods of observation give the following results (1.) by the deflection of the plumbline in the proximity of the Shehallien Mountain (Gaelic, eae) in Perthshire, 4713, as determined by Maskelyne, Hutton, and Playtair (1774-1776 and 1810), according to a method that had been proposed by Newton; (2.) by pendulum vibrations on mountains, 4837 (Carlinis observations on Mount Cenis compared with Biots observations at Bordeaux, Hffemer. Astron. di Milano, 1824, p. 184); G) by the torsion balance used by Cavendish, with an apparatus originally devised by Mitchell, 548 (according to Huttons revision of the calculation, 532, and according to that of Eduard Schmidt, 552; Lehrbuch der Math. Geographie, bd. i., s. 487); by the torsion balance, according to Reich, 544. In the calculation of these experiments of Professor Reich, which have been made with masterly accuracy, the original mean result was 543 (with a probable error of only 0.00233), a result which, being increased by the Loom by which the Earths centrifugal force diminishes the force of gravity for the latitude of Freiberg (50 55), becomes changed to 544. The employment of cast iron instead of lead has not presented any sensible difference, or none exceeding the limits of errors of observation, hence disclosing no traces of magnetic influences. (Reich, Versuche aber die mittlere Dichtighett der Erde, 1838, s. 60, 62, and 66.) By the assumption of too slight a degree of ellipticity of the Earth, and by the uncertainty of the estimations regarding the density of rocks on its surface, the mean density of the Earth, as deduced from experiments on and near mountains, was found about one sixth smaller than it really is, namely, 4761 (Laplace, Mécan. Céleste, t. v., p. 46), or 4785. (Eduard Schmidt, Lehrb. der Math. Geogr., bd. i., 387 und 418.) On Halleyshypothesis of the Earth hollow sphere (noticed in page 171), which was the germ of Franklins ideas concerning earthquakes, see Philos. Trans. for the year 1693, vol. xvii., p. 563 (On the Structure of the Internal Parts of the Earth, and the concave habited Arch of the Shell). Halley regarded it as more worthy of the Creator that the Earth, like a house of several stories, should be inhabited both without and within. For light in the hollow sphere (p. 57) provision might in some manner be contrived.} the last is the most certain, since it is independent of the difficult determination of the density of the mineral masses of which the spherical segment of the mountain consists near which the observations are made. According to the most recent experiments of Reich, the result obtained is 544; that is to say, the mean density of the whole Earth is 544 times greater than that of pure water. As, according to the nature of the mineralogical strata constituting the dry continental part of the Earths surface, the mean density of this portion scarcely amounts to 27, and the density of the dry and liquid surface conjointly to scarcely 16, it follows that the elliptical unequally compressed layers of the interior must greatly increase in density toward the center, either through pressure or owing to the heterogeneous nature of the substances. Here again we see that the vertical, as well as the horizontally vibrating pendulum, may justly be termed a geognostical instrument. The results obtained by the employment of an instrument of this kind have led celebrated physicists, according to the difference of the hypothesis from which they started, to adopt entirely opposite views regarding the nature of the interior of the globe. It has been computed at what depths liquid or even gaseous substances would, from the pressure of their own superimposed strata, attain a density exceeding that of platinum or even iridium; and in order that the compression which has been determined within such narrow limits might be brought into harmony with the assumption of simple and infinitely compressible matter, Leslie has ingeniously conceived the nucleus of the world to be a hollow sphere, filled with an assumed imponderable matter, having an enormous force of expansion. These venturesome and arbitrary conjectures have given rise, in wholly unscientific circles, to still more fantastic notions. The hollow sphere has by degrees been peopled with plants and animals, and two small subterranean revolving planets Pluto and Proserpine were imaginatively supposed to shed over it their mild light; as, however, it was further imagined that an everuniform temperature reigned in these internal regions, the air, which was made selfluminous by compression, might well render the planets of this lower world unnecessary. Near the north pole, at 82 latitude, whence the polar light emanates, was an enormous opening, through which a descent might be made into the hollow sphere, and Sir Humphrey Davy and myself were even publicly and frequently invited by Captain Symmes to enter upon this subterranean expedition so powerful is the morbid inclination of men to fill unknown spaces with shapes of wonder, totally unmindful of the counter evidence furnished by well-attested facts and universally acknowledged natural laws. Even the celebrated Halley, at the end of the seventeenth century, hollowed out the Earth in his magnetic speculations Men were invited to believe that a subterranean freely-rotating nucleus occasions by its position the diurnal and annual changes of magnetic declination. It has thus been attempted in our own day, with tedious solemnity, to clothe in a scientific garb the quaintly-devised fiction of the humorous Holberg.\footnote{[The work referred to, one of the wittiest productions of the learned Norwegian satirist and dramatist Holberg, was written in Latin, and first appeared under the following title Nicolai Klimit iter subterraneum novam telluris theoriam ac historiam quinte monarchia adhuc nobis incognite exhibens e bibliotheca b. Abelini. Hafnie et Lipsie sumt.Jac. Preuss, 1741. An admirable Danish translation of this learned but severe satire on the institutions, morals, and manners of the inhabitants of the upper Earth, appeared at Copenhagen in 1789, and was entitled Niels Kim's underjordiske reise ved Ludwig Holberg, oversat efter den Latinske original af Jens Baggesen. Holberg, who studied for a time at Oxford, was born at Bergen in 1685, and died in 1754 as Rector of the University of Copenhagen.] -- Tr.} 

