
\chapter{Geographical \\Distribution}

\lettrine[lines=4]{\goudy T}{he} consideration of the increase of heat with the increase of depth toward the interior of our planet, and of the reaction of the interior on the external crust, leads us to the long series of volcanic phenomena. These elastic forces are manifested in earthquakes, eruptions of gas, hot wells, mud volcanoes and lava currents from craters of eruptions, and even in producing alterations in the level of the sea.\footnote{[See Daubeney On Volcanoes, 2nd edit., 1848, p. 539, c., on the so-called mud volcanoes, and the reasons advanced in favor of adopting the term salses to designate these phenomena.]-- Tr.} Large plains and variously indented continents are raised or sunk, lands are separated from seas, and the ocean itself, which is permeated by hot and cold currents, coagulates at both poles, converting water into dense masses of rock, which are either stratified and fixed, or broken up into floating banks. The boundaries of sea and land, of fluids and solids, are thus variously and frequently changed. Plains have undergone oscillatory movements, being alternately elevated and depressed. After the elevation of continents, mountain chains were raised upon long fissures, mostly parallel, and, in that case, probably cotemporaneous; and salt lakes and inland seas, long inhabited by the same creatures, were forcibly separated, the fossil remains of shells and zoophytes still giving evidence of their original connection. Thus, in following phenomena in their mutual dependence, we are led from the consideration of the forces acting in the interior of the Earth to those which cause eruptions on its surface, and by the pressure of elastic vapors give rise to burning streams of lava that flow from open fissures.

The same powers that raised the chains of the Andes and the Himalaya to the regions of perpetual snow have occasioned new compositions and new textures in the rocky masses and have altered the strata which had been previously deposited from fluids impregnated with organic substances. We here trace the series of formations, divided and superposed according to their age, and depending upon the changes of configuration of the surface, the dynamic relations of upheaving forces, and the chemical action of vapors issuing from the fissures.

The form and distribution of continents, that is to say, of that solid portion of the Earth's surface which is suited to the luxurious development of vegetable life, are associated by intimate connection and reciprocal action with the encircling sea, in which organic life is almost entirely limited to the animal world. The liquid element is again covered by the atmosphere, an aerial ocean in which the mountain chains and high plains of the dry land rise like shoals, occasioning a variety of currents and changes of temperature, collecting vapor from the region of clouds, and distributing life and motion by the action of the streams of water which flow from their declivities.

While the geography of plants and animals depends ontheze intricate relations of the distribution of sea and land, theconfiguration of the surface, and the direction of isothermallines (or zones of equal mean annual heat), we find that thecase is totally different when we consider the human racethe last and noblest subject in a physical description of theglobe. The characteristic differences in races, and their relative numerical distribution over the Earths surface, are conditions aflected not by natural relations alone, but at the sametime and specially, by the progress of civilization, and by moraland intellectual cultivation, on which depends the politicalsuperiority that distinguishes national progress. Some fewraces, clinging, as it were, tv the soil, are supplanted and ruinedby the dangerous vicinity of others more civilized than themselves, until scarce a traceof their existence remains. Otherraces, again, not the strongest in numbers, traverse the liquidelement, and thus become the first to acquire, although late,a geographical knowledge of at least the maritime lands of thewhole surface of our globe, from pole to pole.

I have thus, before we enter on the individual characters of that portion of the delineation of nature which includes the sphere of telluric phenomena, shown generally in what manner the consideration of the form of the Earth and the incessant action of electromagnetism and subterranean heat may enable us to embrace in one view the relations of horizontal expansion and elevation on the Earth's surface, the geognostic type of formations, the domain of the ocean (of the liquid portions of the Earth), the atmosphere with its meteorological processes, the geographical distribution of plants and animals, and, finally, the physical gradations of the human race, which is, exclusively and everywhere, susceptible to intellectual culture. This unity of contemplation presupposes a connection of phenomena according to their internal combination. A mere tabular arrangement of these facts would not fulfill the object I have proposed to myself, and would not satisfy that requirement for cosmical presentation awakened in me by the aspect of nature in my journeyings by sea and land, by the careful study of forms and forces, and by a vivid impression of the unity of nature in the midst of the most varied portions of the Earth. In the rapid advance of all branches of physical science, much that is deficient in this attempt will, perhaps, at no remote period, be corrected and rendered more perfect, for it belongs to the history of the development of knowledge that portions which have long stood isolated become gradually connected and subject to higher laws. I only indicate the empirical path, in which I and many others of similar pursuits with myself are advancing, full of expectation that, as Plato tells us Socrates once desired, Nature may be interpreted by reason alone.\footnote{Plato, Phedo, p.97. (Arist., Metaph., p. 985.) Compare Hegel,Philosophie der Geschichte, 1840, s. 16.}
