
\chapter{Aurora Borealis}
    

\lettrine[lines=4]{\goudy T}{}errestrial magnetism, and the electrodynamic forces computed by the intellectual Amp\'{e}re,\footnote{Instead of ascribing the internal heat of the Earth to the transition of matter from a vaporlike fluid to a solid condition, which accompanies the formation of the planets, Amp\'{e}re has provided the idea, which I regard as highly improbable, that the Earth's temperature may be the consequence of the continuous chemical action of a nucleus of the metals of the earths and alkalies on the oxidizing external crust. It cannot be doubted, he observes in his masterly Th\'{e}orie des Ph\'{e}nom\'{e}nes Electrodynamiques, 1826, p. 199, that electromagnetic currents exist in the interior of the globe, and that these currents are the cause of its temperature. They arise from the action of a central metallic nucleus, composed of the metals discovered by Sir Humphrey Davy, acting on the surrounding oxidized layer.} stand in simultaneous and intimate connection with the terrestrial or polar light, as well as with the internal and external heat of our planet, whose magnetic poles may be considered as the poles of cold.\footnote{The remarkable connection between the curvature of the magnetic lines and that of my isothermal lines was first detected by Sir David Brewster. See the Transactions of the Royal Society of Edinburgh, vol. ix., 1821, p. 318, and Treatise on Magnetism, 1837, p. 42, 44, 47, and 268. This distinguished physicist admits two cold poles (poles of maximum Cold) in the northern hemisphere, an American one near Cape Walker (73 lat., 100 W. long.), and an Asiatic one (73 lat., 80 E. long.); whence arise, according to him, two hot and two cold meridians, i.e., meridians of greatest heat and cold. Even in the sixteenth century, Acosta (Historia Natural de las Indias, 1589, lib. i., cap. 17), grounding his opinion on the observations of a very experienced Portuguese pilot, taught that there were four lines without declination. It would seem from the controversy of a Bond (the author of The Longitude Found, 1676) with Beckborrow, that this view in some measure influenced Halley in his theory of four magnetic poles. See my Examen Critique de l'Hist. de la G\'{e}ographie, t. iii., p. 60.} The old conjecture hazarded one hundred and twenty-eight years since by Halley,\footnote{Halley, in the Philosophical Transactions, vol. xxix. (for 1714-1716, No. 341.} that the Aurora Borealis was a magnetic phenomenon, has acquired empirical certainty from Faraday's brilliant discovery of the evolution of light by magnetic forces. The northern light is preceded by premonitory signs. Thus, in the morning before the occurrence of the phenomenon, the irregular horary course of the magnetic needle generally indicates a disturbance of the equilibrium in the distribution of terrestrial magnetism.\footnote{[The Aurora Borealis of October 24th, 1847, which was one of the most brilliant ever known in this country, was preceded by great magnetic disturbance. On the 22nd of October the maximum of the west declination was 23 10; on the 23rd the position of the magnet was continually changing, and the extreme west declinations were between 220 44 and 23 37; on the night between the 23rd and 24th of October, the changes of position were very large and very frequent, the magnet at times moving across the field so rapidly that a difficulty was experienced in following it. During the day of the 24th of October, there was a constant change of position, but after midnight, when the Aurora began perceptibly to decline in brightness, the disturbance entirely ceased. The changes of position of the horizontal force magnet were as large and as frequent as those of the declination magnet, but the vertical force magnet was at no time so much affected as the other two instruments. See On the Aurora Borealis, as it was seen on Sunday evening, October 24th, 1847, at Blackheath, by James Glaisher, Esq., of the Royal Observatory, Greenwich, in the London, Edinburgh, and Dublin Philos. Mag. and Journal of Science for Nov., 1847. See further, An Account of the Aurora Borealis of October the 24th, 1847, by John H. Morgan, Esq. We must not omit to mention that magnetic disturbance is now registered by a photographic process. The self-registering photographic apparatus used for this purpose in the Observatory at Greenwich was designed by Mr. Brooke, and another ingenious instrument of this kind has been invented by Mr. F. Ronalds, of the Richmond Observatory.] -- Tr.} When this disturbance attains a great degree of intensity, the equilibrium of the distribution is restored by a discharge attended by a development of light.

The Aurora\footnote{Dove, in Poggend., Annalen, bd. xx., 8. 341; bd. xix., s. 388. The declination needle acts in very nearly the same way as an atmospheric electrometer, whose divergence in like manner shows the increased tension of the electricity before this has become so great as to yield a spark. See, also, the excellent observations of Professor Kamtz, in his Lehrbuch der Meteorologie, bd. iii., s. 511-519, and Sir David Brewster, in his Treatise on Magnetism, p. 280. Regarding the magnetic properties of the galvanic flame, or luminous arch from a Bunsen's carbon and zinc battery, see Casselmann's Beobachtungen (Marburg, 1844), s. 56-62.} itself is, therefore, not to be regarded as an externally manifested cause of this disturbance, but rather as a result of telluric activity, manifested on the one side by the appearance of the light, and on the other by the vibrations of the magnetic needle. The splendid appearance of colored polar light is the act of discharge, the termination of a magnetic storm, as in an electrical storm a development of light—the flash of lightning—indicates the restoration of the disturbed equilibrium in the distribution of the electricity. An electric storm is generally confined to a small space, beyond the limits of which the condition of the atmospheric electricity remains unchanged. A magnetic storm, on the other hand, shows its influence on the course of the needle over large portions of continents, and, as Arago first discovered, far from the spot where the evolution of light was visible. It is not improbable that, as heavily charged threatening clouds, owing to frequent transitions of the atmospheric electricity to an opposite condition, are not always discharged, accompanied by lightning, so likewise magnetic storms may occasion far-extending disturbances in the horary course of the needle, without there being any positive necessity that the equilibrium of the distribution should be restored by explosion, or by the passage of luminous effusions from one of the poles to the equator, or from pole to pole.

In collecting all the individual features of the phenomenon in one general picture, we must not omit to describe the origin and course of a perfectly developed Aurora Borealis. Low down in the distant horizon, about the part of the heavens which is intersected by the magnetic meridian, the sky which was previously clear is at once overcast. A dense wall or bank of cloud seems to rise gradually higher and higher, until it attains an elevation of 8 or 10 degrees. The color of the dark segment passes into brown or violet, and stars are visible through the cloudy stratum, as when a dense smoke darkens the sky. A broad, brightly luminous arch, first white, then yellow, encircles the dark segment; but as the brilliant arch appears subsequently to the smoky gray segment, we cannot agree with Argelander in ascribing the latter to the effect of mere contrast with the bright luminous margin.\footnote{Argelander, in the important observations on the northern light embodied in the Vorträge über die physikalisch-ökonomischen Gesellschaft zu Königsberg, bd. i., 1834, s. 257-264.} The highest point of the arch of light is, according to accurate observations made on this subject,\footnote{For an account of the results of the observations of Lottin, Bravais, and Siljerström, who spent a winter at Bosekop, on the coast of Lapland (70 N. lat.), and in 210 nights saw the northern lights 160 times, see the Comptes Rendus de l'Acad. des Sciences, t. x., p. 289, and Martins's Météorologie, 1843, p. 453. See, also, Argelander, in the Vorträge geh. in der Königsberg Gesellschaft, bd. i., s. 259.} not generally in the magnetic meridian itself, but from 5 to 18 toward the direction of the magnetic declination of the place.\footnote{[Prof Challis, of Cambridge, states that in the Aurora of October 24th, 1847, the streamers all converged toward a single point of the heavens, situated in or very near a vertical circle passing through the magnetic pole. Around this point a corona was formed, the rays of which diverged in all directions from the center, leaving a space free from light; its azimuth was 18° 41' from south to east, and its altitude 69° 54'. See Professor Challis, in the Atheneum, Oct. 31, 1847.] -- Tr} In northern latitudes, in the immediate vicinity of the magnetic pole, the smokelike conical segment appears less dark, and sometimes is not even seen. Where the horizontal force is the weakest, the middle of the luminous arch deviates the most from the magnetic meridian.

The luminous arch remains sometimes for hours together flashing and kindling in ever-varying undulations, before rays and streamers emanate from it, and shoot up to the zenith. The more intense the discharges of the northern light, the more bright is the play of colors, through all the varying gradations from violet and bluish white to green and crimson. Even in ordinary electricity excited by friction, the sparks are only colored in cases where the explosion is very violent after great tension. The magnetic columns of flame rise either singly from the luminous arch, blended with black rays similar to thick smoke, or simultaneously in many opposite points of the horizon, uniting together to form a flickering sea of flame, whose brilliant beauty admits of no adequate description, as the luminous waves are every moment assuming new and varying forms. The intensity of this light is at times so great, that Lowen\'{e}rn (on the 29th of June, 1786) recognized the coruscation of the polar light in bright sunshine. Motion renders the phenomenon more visible. Round the point in the vault of heaven which corresponds to the direction of the inclination of the needle, the beams unite together to form the so-called corona, the crown of the northern light, which encircles the summit of the heavenly canopy with a milder radiance and unflickering emanations of light. It is only in rare instances that a perfect crown or circle is formed, but on its completion the phenomenon has invariably reached its maximum, and the radiations become less frequent, shorter, and more colorless. The crown and the luminous arches break up, and the whole vault of heaven becomes covered with irregularly scattered, broad, faint, almost ashy-gray luminous immovable patches, which in their turn disappear, leaving nothing but a trace of the dark, smokelike segment on the horizon. There often remains nothing of the whole spectacle but a white, delicate cloud with feathery edges, or divided at equal distances into small roundish groups like cirrocumuli.

This connection of the polar light with the most delicate airy clouds deserves special attention because it shows that the electromagnetic evolution of light is a part of a meteorological process. Terrestrial magnetism here manifests its influence on the atmosphere and on the condensation of aqueous vapor. The fleecy clouds seen in Iceland by Thienemann, and which he considered to be the northern light, have been seen in recent times by Franklin and Richardson near the American north pole, and by Admiral Wrangel on the Siberian coast of the Polar Sea. All remarked that the Aurora flashed forth in the most vivid beams when masses of cirrus strata were hovering in the upper regions of the air, and when these were so thin that their presence could only be recognized by the formation of a halo round the moon. These clouds sometimes range themselves, even by day, in a similar manner to the beams of the Aurora, and then disturb the course of the magnetic needle in the same manner as the latter. On the morning after every distinct nocturnal Aurora, the same superimposed strata of clouds have still been observed that had previously been luminous.\footnote{John Franklin, Narrative of a Journey to the Shores of the Polar Sea, in the Years 1819-1822, p. 552 and 597; Thienemann, in the Edinburgh Philosophical Journal, vol. xx., p. 336; Farquharson, in vol. vi., p. 392, of the same journal; Wrangel, Phys. Beob., 8. 59. Parry even saw the great arch of the northern light continue throughout the day. (Journal of a Second Voyage, performed in 1821-1823, p. 156.) Something of the same nature was seen in England on the 9th of September, 1827. A luminous arch, 20 high, with columns proceeding from it, was seen at noon in a part of the sky that had been clear after rain. (Journal of the Royal Institution of Great Britain, 1828, Jan., p. 429.)} The apparently converging polar zones (streaks of clouds in the direction of the magnetic meridian), which constantly occupied my attention during my journeys on the elevated plateaus of Mexico and in Northern Asia, belong probably to the same group of diurnal phenomena.\footnote{On my return from my American travels, I described the delicate cirrocumulus cloud, which appears uniformly divided, as if by the action of repulsive forces, under the name of polar bands (bandes polaires), because their perspective point of convergence is mostly at first in the magnetic pole, so that the parallel rows of fleecy clouds follow the magnetic meridian. One peculiarity of this mysterious phenomenon is the oscillation, or occasionally the gradually progressive motion, of the point of convergence. It is usually observed that the bands are only fully developed in one region of the heavens, and they are seen to move first from south to north, and then gradually from east to west. I could not trace any connection between the advancing motion of the bands and alterations of the currents of air in the higher regions of the atmosphere. They occur when the air is extremely calm and the heavens are quite serene, and are much more common under the tropics than in the temperate and frigid zones. I have seen this phenomenon on the Andes, almost under the equator, at an elevation of 15,920 feet, and in Northern Asia, in the plains of Krasnojarski, south of Buchtarminsk, so similarly developed, that we must regard the influences producing it as very widely distributed, and as depending on general natural forces. See the important observations of Kaimtz (Vorlesungen uber Meteorologie, 1840, s. 146), and the more recent ones of Martins and Bravais (Météorologie, 1843, p. 117). In south polar bands, composed of very delicate clouds, observed by Arago at Paris on the 23rd of June, 1844, dark rays shot upward from an arch running east and west. We have missed made mention of black rays, resembling dark smoke, as occurring in brilliant nocturnal northern lights.}

Southern lights have often been seen in England by the intelligent and indefatigable observer Dalton, and northern lights have been observed in the southern hemisphere as far as 45 latitude (as on the 14th of January, 1831). On occasions that are by no means of rare occurrence, the equilibrium at both poles has been simultaneously disturbed. I have discovered with certainty that northern polar lights have been seen within the tropics in Mexico and Peru. We must distinguish between the sphere of simultaneous visibility of the phenomenon and the zones of the Earth where it is seen almost nightly. Every observer no doubt sees a separate Aurora of his own, as he sees a separate rainbow. A great portion of the Earth simultaneously engenders these phenomena of emanations of light. Many nights may be instanced in which the phenomenon has been simultaneously observed in England and in Pennsylvania, in Rome and in Pekin. When it is stated that Auroras diminish with the decrease of latitude, the latitude must be understood to be magnetic, and as measured by its distance from the magnetic pole. In Iceland, in Greenland, Newfoundland, on the shores of the Slave Lake, and at Fort Enterprise in Northern Canada, these lights appear almost every night at certain seasons of the year, celebrating with their flashing beams, according to the mode of expression common to the inhabitants of the Shetland Isles, a merry dance in heaven.\footnote{The northern lights are called by the Shetland Islanders the merry dancers. (Kendal, in the Quarterly Journal of Science, new series, vol. iv., p. 395.)} While the Aurora is a phenomenon of rare occurrence in Italy, it is frequently seen in the latitude of Philadelphia (39 57), owing to the southern position of the American magnetic pole. In the districts which are remarkable, in the New Continent and the Siberian coasts, for the frequent occurrence of this phenomenon, there are special regions or zones of longitude in which the polar light is particularly bright and brilliant.\footnote{See Munckes excellent work in the new edition of Gehlers Physik Werterbuch, bd. vii., i., 8. 113268, and especially s. 158.} The existence of local influences cannot, therefore, be denied in these cases. Wrangel saw the brilliancy diminish as he left the shores of the Polar Sea, about NischneKolymsk. The observations made in the North Polar expedition appear to prove that in the immediate vicinity of the magnetic pole the development of light is not in the least degree more intense or frequent than at some distance from it.

The knowledge which we at present possess of the altitude of the polar light is based on measurements which, from their nature, the constant oscillation of the phenomenon of light, and the consequent uncertainty of the angle of parallax, are not deserving of much confidence. The results obtained, setting aside the older data, fluctuate between several miles and an elevation of 3000 or 4000 feet; and, in all probability, the northern lights at different times occur at very different elevations.\footnote{Farquharson in the Edinburgh Philos. Journal, vol. xvi., p. 304; Philos. Transactions for 1829, p. 113.

[The height of the bow of light of the Aurora seen at the Cambridge Observatory, March 19, 1847, was determined by Professors Challis, of Cambridge, and Chevallier, of Durham, to be 177 miles above the surface of the Earth. See the notice of this meteor in An Account of the Aurora Borealis of Oct. 24, 1847, by John H. Morgan, Esq., 1848.] -- Tr.} The most recent observers are disposed to place the phenomenon in the region of clouds, and not on the confines of the atmosphere; and they even believe that the rays of the Aurora may be affected by winds and currents of air, if the phenomenon of light, by which alone the existence of an electromagnetic current is appreciable, be actually connected with material groups of vesicles of vapor in motion, or, more correctly speaking, if light penetrate them, passing from one vesicle to another. Franklin saw near Great Bear Lake a beaming northern light, the lower side of which he thought illuminated a stratum of clouds, while, at a distance of only eighteen geographical miles, Kendal, who was on watch throughout the whole night, and never lost sight of the sky, perceived no phenomenon of light. The assertion, so frequently maintained of late, that the rays of the Aurora have been seen to shoot down to the ground between the spectator and some neighboring hill, is open to the charge of optical delusion, as in the cases of strokes of lightning or of the fall of fireballs.

Whether the magnetic storms, whose local character we have illustrated by such remarkable examples, share noise as well as light in common with electric storms, is a question that has become difficult to answer, since implicit confidence is no longer yielded to the relations of Greenland whale fishers and Siberian fox hunters. Northern lights appear to have become less noisy since their occurrences have been more accurately recorded. Parry, Franklin, and Richardson, near the north pole; Thienemann in Iceland; Gieseke in Greenland; Letuc and Bravais, near the North Cape; Wrangel and Anjou, on the coast of the Polar Sea, have together seen the Aurora thousands of times, but never heard any sound attending the phenomenon. If this negative testimony should not be deemed equivalent to the positive counter evidence of Hearne on the mouth of the Copper River and of Henderson in Iceland, it must be remembered that, although Hood heard a noise as of quickly moved musket balls and a slight cracking sound during an Aurora, he also noticed the same noise on the following day, when there was no northern light to be seen; and it must not be forgotten that Wrangel and Gieseke were fully convinced that the sound they had heard was to be ascribed to the contraction of the ice and the crust of the snow on the sudden cooling of the atmosphere. The belief in a crackling sound has arisen, not among the people generally, but rather among learned travelers, because in earlier times the northern light was declared to be an effect of atmospheric electricity, on account of the luminous manifestation of the electricity in rarefied space, and the observers found it easy to hear what they wished to hear. Recent experiments with very sensitive electrometers have hitherto, contrary to the expectation generally entertained, yielded only negative results. The condition of the electricity in the atmosphere\footnote{[Mr. James Glaisher, of the Royal Observatory, Greenwich, in his interesting Remarks on the Weather during the Quarter ending December 31st, 1847, says, It is a fact well worthy of notice, that from the beginning of this quarter till the 20th of December, the electricity of the atmosphere was almost always in a neutral state, so that no signs of electricity were shown for several days together by any of the electrical instruments. During this period there were eight exhibitions of the Aurora Borealis, of which one was the peculiarly bright display of the meteor on the 24th of October. These frequent exhibitions of brilliant Aurore seem to depend upon many remarkable meteorological relations, for we find, according to Mr. Glaisher's statement in the paper to which we have already alluded, that the previous fifty years afford no parallel season to the closing one of 1847. The mean temperature of evaporation and of the dew point, the mean elastic force of vapor, the mean reading of the barometer, and the mean daily range of the readings of the thermometers in dir, were all greater at Greenwich during that season of 1847 than the average range of many preceding years.] -- Tr.} is not found to be changed during the most intense Aurora; but, on the other hand, the three expressions of the power of terrestrial magnetism, declination, inclination, and intensity, are all affected by polar light, so that in the same night, and at different periods of the magnetic development, the same end of the needle is both attracted and repelled. The assertion made by Parry, on the strength of the data yielded by his observations in the neighborhood of the magnetic pole at Melville Island, that the Aurora did not disturb, but rather exercised a calming influence on the magnetic needle, has been satisfactorily refuted by Parry's own more exact researches,\footnote{Kamtz, Lehrbuch der Meteorologie, bd. iii., s. 498 und 501.2} detailed in his journal, and by the admirable observations of Richardson, Hood, and Franklin in Northern Canada, and lastly by Bravais and Lottin in Lapland. The process of the Aurora is, as has already been observed, the restoration of a disturbed condition of equilibrium. The effect on the needle is different according to the degree of intensity of the explosion. It was only unappreciable at the gloomy winter station of Bosekop when the phenomenon of light was very faint and low in the horizon. The shooting cylinders of rays have been aptly compared to the flame which rises in the closed circuit of a voltaic pile between two points of carbon at a considerable distance apart, or, according to Fizeau, to the flame rising between a silver and a carbon point, and attracted or repelled by the magnet. This analogy certainly sets aside the necessity of assuming the existence of metallic vapors in the atmosphere, which some celebrated physicists have regarded as the substratum of the northern light.

When we apply the indefinite term polar light to the luminous phenomenon which we ascribe to a galvanic current, that is to say, to the motion of electricity in a closed circuit, we merely indicate the local direction in which the evolution of light is most frequently, although by no means invariably, seen. This phenomenon derives the greater part of its importance from the fact that the Earth becomes self-luminous, and that as a planet, besides the light which it receives from the central body, the Sun, it shows itself capable in itself of developing light. The intensity of the terrestrial light, or, rather, the luminosity which is diffused, exceeds, in cases of the brightest colored radiation toward the zenith, the light of the Moon in its first quarter. Occasionally, as on the 7th of January, 1831, printed characters could be read without difficulty. This almost uninterrupted development of light in the Earth leads us by analogy to the remarkable process exhibited in Venus. The portion of this planet which is not illumined by the Sun often shines with a phosphorescent light of its own. It is not improbable that the Moon, Jupiter, and the comets shine with an independent light, besides the reflected solar light visible through the polariscope. Without speaking of the problematical but yet ordinary mode in which the sky is illuminated, when a low cloud may be seen to shine with an uninterrupted flickering light for many minutes together, we still meet with other instances of terrestrial development of light in our atmosphere. In this category we may reckon the celebrated luminous mists seen in 1783 and 1831; the steady luminous appearance exhibited without any flickering in great clouds observed by Rozier and Beccaria; and lastly, as Arago\footnote{Arago, on the dry fogs of 1783 and 1831, which illuminated thenight, in the Annuaire du Bureau des Longitudes, 1832, p.246 and 2503and, regarding extraordinary luminous appearances in clouds withoutstorms, see Notices sur la Tonnerre, in the Annuaire pour Van. 1838,p. 279-285.} well remarks, the faint diffused light which guides the steps of the traveler in cloudy, starless, and moonless nights in autumn and winter, even when there is no snow on the ground. As in polar light or the electromagnetic storm, a current of brilliant and often colored light streams through the atmosphere in high latitudes, so also in the torrid zones between the tropics, the ocean simultaneously develops light over a space of many thousand square miles. Here the magical effect of light is owing to the forces of organic nature. Foaming with light, the eddying waves flash in phosphorescent sparks over the wide expanse of waters, where every scintillation is the vital manifestation of an invisible animal world. So varied are the sources of terrestrial light. Must we still suppose this light to be latent, and combined in vapors, in order to explain Moser's images produced at a distance—a discovery in which reality has hitherto manifested itself like a mere phantom of the imagination.
    