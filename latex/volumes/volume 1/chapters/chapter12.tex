\chapter{Terrestrial Phenomena}

\lettrine[lines=4]{\goudy F}{rom} the regions of celestial forms, the domain of Uranus, we will now descend to the more contracted sphere of terrestrial forces, to the interior of the Earth itself. A mysterious chain links together both classes of phenomena. According to the ancient signification of the Titanic myth,\footnote{Otfried Miller, Prolegomena, 8. 373.} the powers of organic life, that is to say, the great order of nature, depend upon the combined action of heaven and earth. If we suppose that the Earth, like all the other planets, originally belonged, according to its origin, to the central body, the Sun, and to the solar atmosphere that has been separated into nebulous rings, the same connection with this contiguous Sun, as well as with all the remote suns that shine in the firmament, is still revealed through the phenomena of light and radiating heat. The difference in the degree of these actions must not lead the physicist, in his delineation of nature, to forget the connection and the common empire of similar forces in the universe. A small fraction of telluric heat is derived from the regions of universal space in which our planetary system is moving, whose temperature (which, according to Fourier, is almost equal to our mean icy polar heat) is the result of the combined radiation of all the stars. The causes that more powerfully excite the light of the Sun in the atmosphere and in the upper strata of our air, that give rise to heat-engendering electric and magnetic currents, and awaken and genially vivify the vital spark in organic structures on the Earth's surface, must be reserved for the subject of our future consideration.

As we purpose for the present to confine ourselves exclusively within the telluric sphere of nature, it will be expedient to cast a preliminary glance over the relations in space of solids and fluids, the form of the Earth, its mean density, and the partial distribution of this density in the interior of our planet, its temperature and its electromagnetic tension. From the consideration of these relations in space, and of the forces inherent in matter, we shall pass to the reaction of the interior on the exterior of our globe; and to the special consideration of a universally distributed natural power - subterranean heat; to the phenomena of earthquakes, exhibited in unequally expanded circles of commotion, which are not referable to the action of dynamic laws alone; to the springing forth of hot wells; and, lastly, to the more powerful actions of volcanic The crust of the Earth, which may scarcely have been perceptibly elevated by the sudden and repeated, or almost uninterrupted shocks by which it has been moved from below, undergoes, nevertheless, great changes in the course of centuries in the relations of the elevation of solid portions, when compared with the surface of the liquid parts, and even in the form of the bottom of the sea. In this manner simultaneous temporary or permanent fissures are opened, by which the interior of the Earth is brought in contact with the external atmosphere. Molten masses, rising from an unknown depth, flow in narrow streams along the declivity of mountains, rushing impetuously onward, or moving slowly and gently, until the fiery source is quenched in the midst of exhalations, and the lava becomes incrusted, as it were, by the solidification of its outer surface. New masses of rocks are thus formed before our eyes, while the older ones are in their turn converted into other forms by the greater or lesser agency of Plutonic forces. Even where no disruption takes place the crystalline molecules are displaced, combining to form bodies of denser texture. The water presents structures of a totally different nature, as, for instance, concretions of animal and vegetable remains, of earthy, calcareous, or aluminous precipitates, agglomerations of finely pulverized mineral bodies, covered with layers of the silicious shields of infusoria, and with transported soils containing the bones of fossil animal forms of a more ancient world. The study of the strata which are so differently formed and arranged before our eyes, and of all that has been so variously dislocated, contorted, and upheaved, by mutual compression and volcanic force, leads the reflective observer, by simple analogies, to draw a comparison between the present and an age that has long passed. It is by a combination of actual phenomena, by an ideal enlargement of relations in space, and of the amount of active forces, that we are able to advance into the long sought and indefinitely anticipated domain of geognosy, which has only within the last half century been based on the solid foundation of scientific deduction.


It has been acutely remarked that, notwithstanding our continual employment of large telescopes, we are less acquainted with the exterior than with the interior of other planets, excepting, perhaps, our own satellite. They have been weighed, and their volume measured; and their mass and density are becoming known with constantly increasing exactness; thanks to the progress made in astronomical observation and calculation. Their physical character is, however, hidden in obscurity, for it is only in our own globe that we can be brought in immediate contact with all the elements of organic and inorganic creation. The diversity of the most heterogeneous substances, their admixtures and metamorphoses, and the ever-changing play of the forces called into action, afford to the human mind both nourishment and enjoyment, and open an immeasurable field of observation, from which the intellectual activity of man derives a great portion of its grandeur and power. The world of perceptive phenomena is reflected in the depths of the ideal world, and the richness of nature and the mass of all that admits of classification gradually become the objects of inductive reasoning.

I would here allude to the advantage of which I have already spoken, possessed by that portion of physical science whose origin is familiar to us, and is connected with our earthly existence. The physical description of celestial bodies, from the remotely glimmering nebulae with their suns, to the central body of our own system, is limited, as we have seen, to general conceptions of the volume and quantity of matter. No manifestation of vital activity is there presented to our senses. It is only from analogies, frequently from purely ideal combinations, that we hazard conjectures on the specific elements of matter, or on their various modifications in the different planetary bodies. But the physical knowledge of the heterogeneous nature of matter, its chemical differences, the regular forms in which its molecules combine together, whether in crystals or granules; its relations to the deflected or decomposed waves of light by which it is penetrated; to radiating, transmitted, or polarized heat; and to the brilliant or invisible, but not, on that account, less active phenomena of electromagnetism - all this inexhaustible treasure, by which the enjoyment of the contemplation of nature is so much heightened, is dependent on the surface of the planet which we inhabit, and more on its solid than on its liquid parts. I have already remarked how greatly the study of natural objects and forces, and the infinite diversity of the sources they open for our consideration, strengthen the mental activity, and call into action every manifestation of intellectual progress. These relations require, however, as little comment as that concatenation of causes by which particular nations are permitted to enjoy a superiority over others in the exercise of a material power derived from their command of a portion of these elementary forces of nature.

If, on the one hand, it were necessary to indicate the difference existing between the nature of our knowledge of the Earth and of that of the celestial regions and their contents, I am no less desirous, on the other hand, to draw attention to the limited boundaries of that portion of space from which we derive all our knowledge of the heterogeneous character of matter. This has been somewhat inappropriately termed the Earth's crust; it includes the strata most contiguous to the upper surface of our planet, and which have been laid open before us by deep fissure-like valleys, or by the labors of man, in the bores and shafts formed by miners. These labors \footnote{In speaking of the greatest depths within the Earth reached by human labor, we nrist recollect that there is a difference between the absolute depth (that is to say, the depth below the Earths surface at that point) and the relative depth (or that beneath the level of the sea). Thegreatest relative depth that man has hitherto reached is probably thepare at the new saltworks at Minden, in Prussia in June, 1844, itwas exactly 1993 feet, the absolute depth being 2231 feet. The temperature of the water at the bottom was 91 F., which, assuming themean temperature of the air at 4993, gives an augmentation of temperature of 1 for every 54 feet. The absolute depth of the Artesianwell of Grenelle, near Paris, is only 1795 feet. According to the account of the missionary Imbert, the firesprings,  Hotsing, of the Chinese, which are sunk to.obtain carbureted  hydrogen gas for saltboiling, far exceed our Artesian springs in depth. In the Chinese provinceof Sztitschuan these firesprings are very commonly of the depth ofmore than 2000 feet; indeed, at Tseulieutsing (the place of continualflow ) there is a Hotsing which, in the year 1812, was found to be 3197feet deep. (Humboldt, Asie Centrale, t. ii., p.521 and 525. Annalesde l Association de la Propagation de la Foi, 1829, No. 16, p. 369.)The relative depth reached at Mount Massi, in Tuscany, south ofVolterra, amounts, according to Matteuci, to only 1253 feet. The boring at the new saltworks near Minden is probably of about the samerelative depth as the coalmine at Apendale, near NewcastleunderLyme, in Staffordshire, where men work 725 yards below the surfaceof the earth. (Thomas Smith, Miners Guide, 1836,p.160.) Unfortunately, I do not know the exact height of its mouth above the levelof the sea. The relative depth of the Monkwearfhouth mine, nearNewcastle, is only 1496 feet. (Phillips, in the Philos. Mag., vol. v.,1834, p. 446.) That of the Liege coalmine, Esp\'{e}rance, at Seraing,is 1355 feet, according to M. von Dechen, the director; and the oldmine of Marihaye, near ValSt.Lambert, in the valley of the Maes,is, according to M. Gernaert, Ing\'{e}nieur des Mines, 1233 feet in depth.The works of greatest absolute depth that have ever been formedare for the most part situated in such elevated plains or valleys thatthey either do not descend so low as the level of the sea, or at mostreach very little below it. Thus the Eselschacht, at Kuttenberg, in Behemia, a mine which can not now be worked, had the enormous absolute depth of 3778 feet. (Fr. A. Schmidt, Berggesetze der dster Mon.,abth. i., bd.i.,s.xxxii.) Also, at St. Daniel and at Geish, on the R\'{e}rerbthel, in the Landgerich\'{e} (or provincial district) of Kitzbithl, therewere, in the sixteenth century, excavations of 3107 feet. The plansof the works of the R\'{e}rerbthel are still preserved. (See Joseph vonSperges, T'yroler Bergwerksgeschichte, 8.121. Compare, also, Humboldt, Gutachten aber Herantreibung des Meissner Stollens in die Freiberger Erzrevier, printed in Herder, aber den jetz begonnenen Erbstollen, 1838, 8. cxxiv.) We may presume that the knowledge of the extraordinary depth of the R\'{e}rerbtihel reached England at an early period,for I find it remarked in Gilbert, de Magnete, that men have penetrated2400 or even 3000 feet into the crust of the Earth. ( Exigua videtarterre portio, que unquam hominibus spectanda emerget aut eruitur;cum profundius in ejus viscera, ultra  florescentis extremitatis corruptelam, aut propter aquas in magnis fodin, tanquam per venas scaturientesaut propter aeris salubrioris ad vitam o erariorum sustinendam necessaril defectum, aut propter ingentes sumptus ad tantos labores exantJandos multasqu difficultates, ad profundiores terre partes penetrara non possumus; adeo ut quadringentas aut  quod rarissime quingentasorgyas in quibusdam metallis descendisse, stupendus omnibus videaturconatus.Gulielmi Gilberti, Colcestrensis, de Magnete Physiologianova. Lond., 1600, p. 40.)} do not extend beyond the leve. of the sea, and zonsequently only about 5,'5;th of theEarths radius. The crystalline masses that have been erupted from active volcanoes, and are generally similar to therocks on the upper surface, have come from depths which,although not accurately determined, must certainly be sixtytimes greater than those to which human labor has been enabled to penetrate. We are able to give in numbers the depthof the shaft where the strata of coal, after penetrating a certain way, rise again at a distance that admits of being accurately defined by measurements. These dips show that thecarboniferous strata, together with the fossil organic remainswhich they contain, must lie, as, for, instance, in Belgium,more than five or six thousand feet\footnote{Basinshaped curved strata, which dip and reappear at measurabledistances, although their deepest portions are beyond the reach of theminer, afford sensible evidence of the nature of the earths crust at greatdepths below its surface. Testimony of this kind possesses, consequently, a great gergnostic interest. Iam indebted to that excellent geognosist, Von Dechen, for the following obseMations. The depth ofthe coal basin of Liege, at Mont St. Gilles, which I, in conjunction withour friend Von Oeynhausen, have ascertained to be 3890 feet belowthe surface, extends 3464 feet below the surface of the sea, for the absolute height of Mont St. Gilles certainly does not much exceed 400feet; the coal basin of Mons is fully 1865 feet deeper. But all thesedepths are trifling compared with those which are presented by thecoal strata of SaarRevier (Saarbritcken). I have found, after repeatedexaminations, that the lowest coal stratum which is known in the neighboarhood of Duttweiler, near Bettingen, northeast of Saarlouis, must descend tc depths of 20,682 and 22,015 feet (or 36 geographical miles)below.the level of the sea. This result exceeds, hy more than 8000feet, the assumption made in the text regarding the basin of the Devonian strata. This coalfield is therefore sunk as far below the surface of the sea as Chimborazo is elevated above itat a depth at whichthe Earths temperature must be as high as 435 F. Hence, from thehighest pinnacles of the Himalaya to the lowest basins containing thevegetation of an earlier world, there is a vertical distance of abont48 000 feet, or of the 435th part of the Earths radius.} below the present level of the sea, and that the calcareous and the curved strata ofthe Devonian basin penetrate twice that depth. If we compare these subterranean basins with the summits of mountainsthat have hitherto been considered as the most elevated portions of the raised crust of the Earth, we obtain a distance of37,000 feet (about seven miles), that is, about the 34;th ofthe Earths radius. These, therefore, would be the limits ofvertical depth and of the superposition of mineral strata towhich geognostical inquiry could penstrate, even if the general elevation of the upper surface of the earth were equal tothe height of the Dhawalagiri n the Himalaya, or of theSorata in Bolivia. All that lies at a greater depth below thelevel of the sea than the shafts or the basins of which I havespoken, the limits to which mans labors have penetrated, o1than the depths to which the sea has in some few instancesbeen sounded (Sir James Ross was unable to find bottom with27,600 feet of line), is as much unknown to us as the interiorof the other planets of our solar system. We only know themass of the whole Earth and its mean density by comparingit with the open strata, which alone are accessible to us. Inthe interior of the Earth, where all knowledge of its chemicaland mineralogical character fails, we are again limited to aspure conjecture, as in the remotest bodies that revolve roundthe Sun. We can determine nothing with certainty regard ing the depth at which the geological strata must be supposedto be in state of softening or of liquid fusion, of the cavitiesoccupied by elastic vapor, of the condition of fluids whenheated under an enormous pressure, or of the law of the inerease of density from the upper surface to the center of the Earth.