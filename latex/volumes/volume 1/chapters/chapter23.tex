
\chapter{Gaseous Emanations}

\lettrine[lines=4]{\goudy H}{aving} thus taken a general view of the activity -- the inner life, as it were -- of the Earth, in respect to its internal heat, its electro-magnetic tension, its emanation of light at the poles, and its irregularly-recurring phenomena of motion, we will now proceed to the consideration of the material products, the chemical changes in the earth’s surface, and the composition of the atmosphere, which are all dependent on planetary vital activity. We see issue from the ground steam and gaseous carbonic acid, almost always free from the admixture of nitrogen ;\footnote{Bischof's comprehensive work, Warmelehre des inneren Erdkérpers.} carbureted hydrogen gas, which has been used in the Chinese province Sse-tschuant\footnote{On the Artesian fire-springs (Ho-tsing) in China, and the ancient use of portable gas (in bamboo canes) in the city of Khiung-tsheu, see Klaproth, in my Asie Genérale, t. iii., p. 519-530.} for several thousand years, and recently in the village of Fredonia, in the State of New York, United States, in cooking and for illumination ; sulphureted hydrogen gas and sulphurous vapors; and, more rarely,\footnote{Boussingault (Annales de Chimie, t. lii., p. 181) observed no evolution of hydrochloric acid from the volcanoes of New Granada, while Monticelli found it in enormous quantity in the eruption of Vesuvius in 1813.} sulphurous and hydrochloric acids.\footnote{[Of the gaseous compounds of sulphur, one, sulphurous acid, appears to predominate chiefly in volcanoes possessing a certain degree of activity, while the other, sulphureted hydrogen, has been most frequently perceived among those in dormant condition. The occurrence of abundant exhalations of sulphuric acid, which have been hitherto noticed chiefly in extinct volcanoes, as, for instance, in a stream issuing from that of Puracé, between Bogota and Quito, from extinct volcanoes in Java, is satisfactorily explained in a recent paper by M. Dumas, Annales de Chimie, Dec., 1846. He shows that when sulphureted hydrogen, at a temperature above 100° Fahr., and still better when near 190°, comes in contact with certain porous bodies, a catalytic action is set up, by which water, sulphuric acid, and sulphur are produced. Hence probably the vast deposits of "sulphur, associated with sulphates of lime and strontian, which are met with in the western parts of Sicily. ] -- Tr.} Such effusions from the fissures of the earth not only occur in the districta of still burning or long-extinguished volcanoes, but they may likewise be observed occasionally in districts where neither trachyte nor any other volcanic rocks are exposed on the earth’s surface. In the chain of Quindiu I have seen sulphur deposited in mica slate from warm sulphurous vapor at an elevation of 6832 feet\footnote{Humboldt, Recueil d@’ Observ. Astronomiques, t. i., p. 311 (Nivellement Barométrique de la Cordillére des Andes, No. 206).} above the level of the sea, while the same species of rock, which was formerly regarded as primitive, contains, in the Cerro Cuello, near Tisean, south of Quito, an immense deposit of sulphur imbedded in pure quartz.

Exhalations of carbonic acid (mofettes) are even in our days
to be considered as the most important of all gaseous emanations, with respect to their number and the amount of their effusion. We see in Germany, in the deep valleys of the Eifel, in the neighborhood of the Lake of Laach,\footnote{[The Lake of Laach, in the district of the Eifel, is an expanse of water two miles in circumference. The thickness of the vegetation on the sides of its crater-like basin renders it difficult to discover the nature of the subjacent rock, but it is probably composed of black cellular augitic lava. The sides of the crater present numerous loose masses, which appear to have been ejected, and consist of glassy feldspar, ice-spar, sodalite, hauyne, spinellane, and leucite. The resemblance between these products and the masses formerly ejected from Vesuvius is most remarkable. (Daubeney On Volcanoes, p. 81.) Dr. Hibbert regards the Lake of Laach as formed in the first instance by a crack caused by the cooling of the crust of the earth, which was widened afterward into a circular cavity by the expansive force of elastic vapors. See History of the Extinct Volcanoes of the Basin of Newwied, 1832.] -- Tr.} in the crater-like valley of the Wehr and in Western Bohemia, exhalations of carbonic acid gas manifest themselves as the last efforts of volcanic activity in or near the foci of an earlier world. In those earlier periods, when a higher terrestrial temperature existed, and when a great number of fissures still remained unfilled, the processes we have described acted more powerfully, and carbonic acid and hot steam were mixed in larger quantities in the atmosphere, from whence it follows, as Adolph Brongniart has ingeniously shown,\footnote{Adolph Brongniart, in the Annales des Sciences Naturelles, t. xv. p- 225.} that the primitive vegetable world must have exhibited almost everywhere, and independently of geographical position, the most luxurious abundance and the fullest development of organism. In these constantly warm and damp atmospheric strata, saturated with carbonic acid, vegetation must have attained a degree of vital activity, and derived the superabundance of nutrition necessary to furnish materials for the formation of the beds of lignite (coal), constituting the inexhaustible means on which are based the physical power and prosperity of nations. Such masses are distributed in basins over certain parts of Europe, occurring in large quantities in the British Islands, in Belgium, in France, in the provinces of the Lower Rhine, and in Upper Silesia. At the same primitive period of universal volcanic activity, those enormous quantities of carbon must also have escaped from the earth which are contained in limestone rocks, and which, if separated from oxygen and reduced to a solid form, would constitute about the eighth part of the absolute bulk of these mountain masses.\footnote{Bischof, op. cit., s. 324, Anm. 2.} That portion of the carbon which was not taken up by alkaline earths, but remained mixed with the atmosphere, as carbonic acid, was gradually consumed by the vegetation of the earlier stages of the world, so that the atmosphere, after being purified by the processes of vegetable life, only retained the small quantity which it now possesses, and which is not injurious to the present organization of animal life. Abundant eruptions of sulphurous vapor have occasioned the destruction of the species of mollusca and fish which inhabited the inland waters of the earlier world, and have given rise to the formation of the contorted beds of gypsum, which have doubtless been frequently affected by shocks of earthquakes.

Gaseous and liquid fluids, mud, and molten earths, ejected from the craters of volcanoes, which are themselves only a kind of intermittent springs, rise from the earth under precisely analogous physical relations.\footnote{Humboldt, Asie Centrale, t. i., p. 43.} All these substances owe their temperature and their chemical character to the place of their origin. The mean temperature of aqueous springs is less than that of the air at the point whence they emerge, if the water flow from a height; but their heat increases with the depth of the strata with which they are in contact at their origin. We have already spoken of the numerical law regulating this increase. The blending of waters that have come from the height of a mountain with those that have sprung from the depths of the earth, render it difficult to determine the position of the \emph{isogeothermal lines}\footnote{On the theory of isogeothermal (chthonisothermal) lines, consult the ingenious labors of Kupifer, in Pogg., Annalen, bd xv., 8. 184, and bd. xxxii., s. 270, in the Voyage dans lOural, p. 382398, and in theEdinburgh Journal of Science, New Series, vol. iv., p. 355. See, also,Kamtz, Lehrb. der Meteor., bd. ii., s. 217; and, on the ascent of thechthonisothermal lines in mountainous districts, Bischof, s. 174198.} (lines of equal internal terrestrial temperature), when this determination is to be made from the temperature of flowing springs. Such, at any rate, is the result I have arrived at from my own observations and those of my fellowtravelers in Northern Asia. The temperature of springs, which has become the subject of such continuous physical investigation during the last half century, depends, like the elevation of the line of perpetual snow, on very many simultaneous and deeply involved causes. It is a function of the temperature of the stratum in which they take their rise, of the specific heat of the soil, and of the quantity and temperature of the meteoric water,\footnote{Leop. v. Buch, in Pogg., Annalen, bd. xii., 8. 405.} which is itself different from the temperature of the lower strata of the atmosphere, according to the different modes of its origin in rain, snow, or hail.\footnote{On the temperature of the drops of rain in Cumana, which fell to72, when the temperature of the air shortly before had been 86 and83, and during the rain sank to 74, see my Relat. Hist., t. ii., p. 22.The raindrops, while falling, change the normal temperature they originally possessed, which depends on the height of the clouds fromwhich they fell, and their heating on their upper surface by the solarrays. The raindrops, on their first production, have a higher temperature than the surrounding medium in the superior strata of our atmosphere, in consequence of the liberation of their latent heat; and theycontinue to rise in temperature, since, in falling through lower andwarmer strata, vapor is precipitated on them, and they thus increase insize (Bischof, Warmelehre des inneren Erdk\'{e}rpers, 8.73); but this additional heating is compensated for by evaporation. The cooling of theair by rain (putting out of the question what probably belongs to theelectric process in storms) is effected by the drops, which are themselves of lower temperature, in consequence of the cold situation inwhich they were formed, and bring down with them a portion of thehigher colder air, and which finally, by moistening the ground, giverise to evaporation. These are the ordinary relations of the phenomenon. When, as occasionally happens, the raindrops are warmer thanthe lower strata of the atmosphere (Humboldt, Rel. Hist., t. iii., p.513), the cause must probably ta sought in higher warmer currents, orin a higher temperature of widelyextended and not very thick clouds,from the action of the suns rays. How, moreover, the phenomenon ofsupplementary rainbows, which are explained by the interference oflight, is connected with the original and increasing size of the fallingdrops, and how an optical phenomenon, if we know how to observe itaccurately, may enlighten us regarding a meteorological process, according to diversity of zone, has been shown, with much talent and in.genuity, by Arago, in the Annuaire for 1836, p. 300.}

Cold springs can only indicate the mean atmospheric temperature when they are unmixed with the waters rising from great depths, or descending from considerable mountain elevations, and when they have passed through a long course at a depth from the surface of the earth which is equal in our latitudes to 40 or 60 feet, and, according to Boussingault, to about one foot in the equinoctial regions;\footnote{The profound investigations of Boussingault fully convince me that in the tropics, the temperature of the ground, at a very slight depth, exactly corresponds with the mean temperature of the air. The following instances are sufficient to illustrate this fact. See Table \ref{tab:temperature}. The doubts about the temperature of the earth within the tropics, of which I am probably, in some degree, the cause, by my observations on the Cave of Caripe (Cueva del Guacharo), Rel. Hist., t. iii., p. 191-196), are resolved by the consideration that I compared the presumed mean temperature of the air of the convent of Caripe, 65.93, not with the temperature of the air of the cave, 65.6, but with the temperature of the subterranean stream, 62.93, although I observed (Rel. Hist., t. iii., p. 146 and 194) that mountain water from a great height might probably be mixed with the water of the cave.} these being the depths at which the invariability of the temperature begins in the temperate and torrid zones, that is to say, the depths at which hourly, diurnal, and monthly changes of heat in the atmosphere cease to be perceived.

\begin{table}[h!]
    \centering
    \caption{Temperature measurements at various stations within tropical zones.}
    \label{tab:temperature}
    \begin{tabular}{m{2cm} m{3cm} m{2cm} m{2cm}}
    \toprule
    \textbf{Stations within Tropical Zones.} & \textbf{Temperature at 1 French foot (1.006 of the English foot) below the earth's surface.} & \textbf{Mean Temperature of the air.} & \textbf{Height, in English feet, above the level of the sea.} \\
    \midrule
    Guayaquil & 78.8 & 78.1 & 0 \\
    Anserma Nuevo & 74.6 & 74.8 & 3444 \\
    Zupia & 70.7 & 70.7 & 4018 \\
    Popayan & 64.7 & 65.6 & 5929 \\
    Quito & 59.9 & 59.9 & 9559 \\
    \end{tabular}
\end{table}